\article{\add[S]{Meeting Policies and Student Voice}}

\add[S]{The provisions in this article exclusively apply to ?.}

\section{Chair and Quorum}
The chair of the \hyperref[USGParliamentDef]{USG Parliament} meetings is the President, and if missing, the Vice-President. A meeting consists of at least a quorum of \add[P]{five} \hyperref[MPDef]{Members of Parliament} which have to include the President or the Vice-President.

\section{Regular Meetings}
Meetings have to take place regularly and are open to the general public of Jacobs University, unless \add[P]{the Parliament needs to discuss confidential agenda items.}
\begin{parenum}
\item Attendance of \hyperref[MPDef]{Members of Parliament} is mandatory.
\item The meetings have to be made public at least two days prior to the meeting.
\item \add[P]{Confidential agenda items are discussed with} elected or appointed members only. The USG Parliament may choose to include any other student \add[P]{if allowed and} considered appropriate by a simple majority. 
\end{parenum}


\section{Decision-Making} 
Each member of the USG shall have one (1) vote in all matters. Decisions shall be made by a simple majority vote with at least fifty percent (50\%) of all members voting. Voting takes place publicly and the names of the "ayes", "nays", and "abstentions" shall be recorded in the minutes. With a simple majority vote the USG may declare a secret vote (ballot) in which case only the number of votes shall be recorded. 


\section{Student Vote}
The USG may ask the \hyperref[studentbody]{USB} via a \hyperref[StudentOpiniondef]{Student Body Opinion} procedure regarding USG decisions and future actions. However, the Student Body Opinion may be overruled with a full unanimity of the \hyperref[USGParliamentDef]{USG Parliament}.

\remove[S]{
\protect\section{External Outreach Power}
The USG shall have the right to communicate with external parties, as stated by Academic Constitution of Jacobs University Bremen. 
\protect\begin{parenum}
\item The USG members, as any student, shall have the right to contact any external bodies, only if they do so as individuals and not representing the USG.
\protect\remove[S]{\protect\item The whole USG \protect\remove[S]{(Parliament and Executive Unit) }has to be informed of any intention for external communication.}\protect\note[Colin]{This is bullshit.}
\item A simple majority vote in the \hyperref[USGParliamentDef]{USG Parliament}, among the USG members or a Student Opinion Decision may enact or prevent any external communication.
\item In particular, the USG President is the main responsible for any external communication.
\end{parenum}

\protect\section{Communication Responsibilities}
One of the USG`s core missions is to inform the \protect\hyperref[studentbody]{USB}. 
\protect\begin{parenum}
    \item A regular task shall be a monthly newsletter of one page with USG past, current and future projects relevant to the respective time-frame.
    \item Failure to do so for two months consecutively constitutes a valid reason for impeachment.
\end{parenum}

\protect\section{Conflicts of Interest}
Conflicts of interest must be declared and the involved members must be omitted from voting. In case of a tie in the voting procedure, the final decision is made jointly by the President, Vice-President and the respective Committee Head.



\protect\section{Resignation and Impeachment}
In the event of an USG Member resigning, not being able, not being willing to continue their tasks as a Member, or being successfully impeached, the \protect\hyperref[USGParliamentDef]{USG Parliament} will appoint an interim for the remaining period of the resigning member.

\protect\begin{parenum}
\item Any Member of the USG may resign provided that the USG is informed two (2) weeks before the intended date of resignation.

\item Any member of the USG shall be impeached by either of the two: Two thirds (2/3) of the (other) elected members or referendum. Such a vote or referendum must clearly state the grounds for impeachment. Repeated \add[S]{unexcused }absence \add[P]{of three (3) }\remove[RS]{consecutive}\add[P]{ meetings or more} and/or not accomplishing designated tasks automatically start the process of impeachment of a member:
\begin{enumerate}
\item The accused member and the impeacher (if any) have one (1) week time to argument and present their case.
\item At the next \hyperref[USGParliamentDef]{USG Parliament} meeting, the impeachment is voted upon.
\end{enumerate}

\item After resignation or impeachment, an officer may not run again for the same office within the term that the resignation \add[P]{or impeachment took place}.
\end{parenum}}
\note[S]{Move to A. GENERAL}