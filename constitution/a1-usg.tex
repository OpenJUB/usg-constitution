\article{The Undergraduate Student Government}
The \ac{USG}\label{USGdef} is formed in order to provide an official and representative organization to receive student questions and suggestions, investigate student problems and take appropriate action, provide the official voice through which the student opinion may be expressed, encourage the development of responsible student participation in the overall policy and decision making processes of the university community, foster an awareness of the students' role in the academic community, enhance the quality and scope of education at Jacobs University Bremen, provide means for responsible and effective participation in the organization of student affairs, and establish guidelines and this Constitution for the Undergraduate Student Government at Jacobs University.\add[S]{The USG is a technocratic parliamentary democracy.}


\section{Executive, Legislative and Judicial Power} 
The \acf{USB}\label{USBdef} shall have the right to self-governance through the \acf{USG}, which is the main authority body of the USB, through:
\begin{enumerate}
    \item the USG \acl{EXE} (composed out of the \hyperref[USGstructure]{Student Committees}), having the executive power of the \hyperref[studentbody]{USB},
    \item the USG \hyperref[USGParliamentDef]{Parliament} \label{USGParliament} (composed out of the \hyperref[ElectedOfficesDef]{Elected Offices}), having the legislative power of the \hyperref[studentbody]{USB},
    \item the USG \hyperref[StudentCourtDef]{Student Court}, having the judicial power.
\end{enumerate}


\section{Power and Duty}
The \hyperref[USGdef]{USG} shall: 
\begin{enumerate}
\item inform the Student Body on recent, present and future endeavours,
\item organize activities involving the \hyperref[studentbody]{USB} and maintain the student budget,
\item establish student committees to deal with specific subjects, define their mandates, appoint their members, and monitor their activities,
\item meet regularly to discuss progress in student committees and to address all issues on the respective agenda,
\item at all times uphold and represent the interests of the \hyperref[studentbody]{USB},
\item enact legislation governing the conduct of the \hyperref[studentbody]{USB} after consultation with the \hyperref[studentbody]{USB},
\item serve as liaison with the university administration, faculty, staff and college authorities.
\end{enumerate}

\section{Student Rights}
Each student shall have the right to:
\begin{enumerate}[nosep] 
\item
vote in popular elections,

\item
\remove[S]{run for elections or appointments for the offices of the\protect \ac{USG}. This includes students currently a part of the \protect\hyperref[USGdef]{USG} at the time of the elections, and excludes students on disciplinary probation,}

\item
apply to be part of a committee,

\item
affiliate themselves with a college of her/his choice, in case of living off-campus,

\item 
affiliate themselves with a focus area of her/his choice, for Foundation Year students,

\item
start a student club and to join any club (regardless of background), according to the requirements set by the Campus Life and the by-laws of the \hyperref[USGdef]{USG},

\item
initiate a popular initiative or petition to suggest legislative proposals or any kind of action to the \hyperref[USGdef]{USG}.
\end{enumerate}

\section{Student Duties}
Each student shall have the duty and obligation to:
\begin{enumerate}
\item
act in accordance with the provisions stipulated by the constitution and the active by-laws that support it,

\item 
accept final decisions of the Disciplinary Board,

\item
 respect decisions taken by the \hyperref[USGdef]{USG} as long as they are in accordance with this constitution.
\end{enumerate}