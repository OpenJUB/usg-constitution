\add[S]{\protect\article{International Affairs Committee By-Laws}
\label{IACByLawsdef}}

\add[S]{
\protect\section{General Responsibilities of the Committee}
The Primary purpose of the international Affairs committee is to create ties with the student government of other universities around the globe, by creating exchange of informations/opportunities in between the USG and the student government of partner universities.
%TIES WITH OTHER UNIVERSITIES SHOULD BE BASED ON THE TARGETING UNIVERSITIES CRITERIA
%Already included in the section about the procedures of the IAC - Colin

\protect\sectionss{Duties and tasks of the Committee}
\protect\begin{parenum}
	\item The main task of the International Affairs Committee is to act as a mediator of
	information from Jacobs University, to other universities and vice versa.
	\item The IAC will work on raising international presence and prestige of Jacobs University.
	\item The IAC will serve to give students more opportunities for small-scale exchange
	programs and projects on a local and international level.
	\item The IAC should serve as a communication channel in order to make students from
	partner universities participate in Jacobs University Events.
	\item The IAC should also develop projects where universities close by (e.g. Hamburg, Bremen), World Track Universities and other universities collaborate with the interaction of our university and other
	universities.
	\item The IAC should also serve as a bridge of communication in order to generate
	collaboration with Clubs/ Societies of Jacobs University and partner universities
	\item The IAC should also embrace collaboration in between different majors of the partner
	universities.
	\item The IAC should also embrace collaboration with Universities in countries of Jacobs
	University Students.
	\item Any student will have the ability to contact or approach the IAC to communicate requests
	or complains and the IAC should proceed with this requests.
	\item The IAC will generate poll in order to follow students preference for partnering
	universities. This universities need to follow the targeting universities criteria
\end{parenum}
	
\protect\section{Procedure and structure of the Committee}
The IAC will have weekly meetings where they discuss relevant projects and opportunities for the student body.
\protect\begin{parenum}
	\item The International Affairs Committee tries to establish links to other universities based on the targetting universities criteria from the International Office.
	\item Get in contact with the target university and generate the link
	\item  Exchange Information about the universities and its events
 \end{parenum}

\protect\section{Structure and subcommittees of the International Affairs Committee}
	The IAC structure should work with the constant interaction of three main sub-committees:
	\protect\begin{enumerate}
		\item The Development Projects Sub-Committee:
		This committee has the obligation to create or follow previous events in order to increase the
		interaction with other universities and its students
		\item The External Expansion Sub-Committee (EEC):
		The EEC is in charge of expanding the IAC alliances, based on the targeting university criteria.
		The EEC should be composed of representatives by mother tongue, that are in charge of expanding the alliances of the IAC, in the regions where the language is spoken.
		\item The Opportunities Exchange Committee (OEC):
		Is OEC is in charge of serving as a bridge of communication with the Jacobs-events organizers and other universities by making information available for Jacobs Students and Partner University Students, aiming to exploit the World Wide opportunities.
		This communication should take place via E-mail and social media.
	\end{enumerate}

\protect\section{Duties of the chair}
The IAC chair has the following obligations:
\protect\begin{parenum}
	\item To have at least monthly meetings with the International Office to support the
	development of the internationalization process.
	\item To coordinate the execution of the different sub-committees.
	\item The IAC chair is in charge of chairing the weekly meetings with the IAC sub
	committees.
	\item The IAC chair will communicate the requests and initiatives of the student body to the
	IO and the sub-committees.
	\item The chair is in charge of recruiting members for the IAC committee from the USB.	
\end{parenum}
}

\add[S]{\protect\section{Composition and Roles}}
\add[S]{The IAC shall be comprised of students of all majors. }

\add[S]{Special roles within the IAC include, but are not limited to:}
\begin{enumerate}
    \item \add[S]{IAC Chair}
    \item \add[S]{IAC Secretary}
\end{enumerate}
