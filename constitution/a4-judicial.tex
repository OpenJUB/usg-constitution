\article{Student Court}
\label{StudentCourtDef}
The \hyperref[StudentCourtDef]{Student Court} shall be an independent body, responsible of checking and judging the legitimacy of the actions of the \hyperref[USGParliamentDef]{USG Parliament} and Executive Unit. As it is may not have regular tasks, it shall be formed upon request by either the \hyperref[studentbody]{USB} via a petition, the \hyperref[USGParliamentDef]{USG Parliament} or the \hyperref[USGexecutiveUnitDef]{USG Executive Unit}, in order to prosecute an accused body.


\section{Power and Duties}
The \hyperref[StudentCourtDef]{Student Court} shall:
\begin{enumerate}
\item at all points of time uphold this constitution in the interests of the USB,
\item interpret this constitution, by-laws, amendments and all subsequent legislation in the spirit
of community standards,
\item mediate between two (2) or more parties of the USG or \hyperref[studentbody]{USB}, at the request of any of
the members involved,
\item determine in a trial, as specified in the by-laws on Student Court procedure, the guilt or
innocence of any object of its jurisdiction who is accused of violations of this constitution
and its by-laws, and assess penalties for such violations according to the by-laws on
Student Court procedure,
\item  be permitted to defer a case to the next higher instance as specified in the by-laws on
Student Court procedure, if the case is decided to be beyond its competence to judge,
\item review the decisions made and actions taken by the Student Parliament and ensure that
they are made under the provisions of this constitution and its by-laws,
\item issue an evaluation of the Student Parliament's budget allocation, efficiency and
progress on agenda points with recommendations for improvement, whenever asked to do so
\item recommend, at its discretion, changes in or additions to this constitution, its by-laws, or
subsequent legislation,
\item be present at official Student Parliament meetings if requested by the Student Parliament
or Government. 
\end{enumerate}

\section{Student Court Formation}
The \hyperref[StudentCourtDef]{Student Court} shall consist of five (5) judges directly elected.
\begin{parenum}
    \item In case the \hyperref[USGParliamentDef]{USG Parliament} is accused of wrong doings, then the \hyperref[USGexecutiveUnitDef]{USG Executive Unit} takes the task of forming the Student Court. Three (3) Judges shall be members of the \hyperref[USGexecutiveUnitDef]{USG Executive Unit}, and two (2) judges shall be elected from the USB.
    
    \item In case the \hyperref[USGexecutiveUnitDef]{USG Executive Unit} is accused of wrong doings, then the \hyperref[USGParliamentDef]{USG Parliament} takes the task of forming the Student Court. Three (3) Judges shall be members of the \hyperref[USGParliamentDef]{USG Parliament}, and two (2) judges shall be elected from the USB.
    
    \item In case both the \hyperref[USGParliamentDef]{USG Parliament} and Executive Unit are accused of wrong doings, then the initiators of the petition against the two bodies takes the task of forming the Student Court. In this case, all five (5) judges shall be elected from the USB.
\end{parenum}



\section{Student Court Procedure}
The \hyperref[StudentCourtDef]{Student Court} shall form within a week of the accusation and shall take the responsibility to reach a decision within two weeks after its formation. 
\begin{enumerate}[label={\textbf{\S\arabic*}}]
    \item A quorum for the meetings shall consist of three (3) Student Court judges and decisions shall be made by simple majority vote.
    \item Each judge shall have one vote.
\end{enumerate}


\section{Conflicts of Interest}
All Student Court judges must declare all possible conflicts of interest to Student Court prior to any hearing. In the interest of impartiality, any judge that has a conflict of interest and also if a judge is involved in a trial before Student Court as the complainant or as the accused, the respective judge shall remove themselves from the case and be replaced.