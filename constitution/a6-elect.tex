\article{Elections}\label{art:elect}
\section{Elections}
\label{electiontimes}
The USG shall hold a general election twice a year, in particular:
\begin{enumerate}
	\item in first week of December for: Focus Area representatives, Members of Parliament  (and for any open position),
	\item in first week of May for: College representatives, Members of Parliament (and for any open position).
\end{enumerate}
The number of Members of Parliament to be elected with each election is determined by the number of open positions in the USG Parliament.

\section{Candidates Eligibility}\label{eligible}
	The student shall be physically present in the respective semester of being in the office.
	\begin{parenum}
		\item Only active members of an USG Executive Committee are eligible to run as a Member of Parliament. 
		\item Students in their second year of study who will spend the fifth semester abroad or will do an internship in their fifth semester may not run for an office in the USG in the May elections of their second year.	
		\item Students in their third year of study or Foundation Year students who will not stay as bachelor students afterwards may not run as Focus area or College representatives in the May elections of their last year.
		\item Foundation Year students may affiliate themselves with a Focus area and are allowed to run as representative for that Focus area. In case they have been offered admission for undergraduate studies and accepted that offer, this must be the Focus area of their future study program.
	\end{parenum}



	\protect\section{Election of the Members of Parliament}\label{MPElectionDef}
	For the election of the Members of Parliament, each student may votes for at most as many candidates as positions to be filled. Afterwards, the Members of Parliament are selected based on the outcome of the election as follows:  
	\protect\begin{parenum}
		\item Firstly, the candidates with most votes (except for a number of candidates equal to the number of Executive Committees) become Members of Parliament. 
		%Firstly, the $5-n$ candidates with most votes (where $n$ denotes the number of executive Committees of the USG) become Member of Parliament. For each Parliament position to be refilled (due ot resignations of impeachments) one additional candidate is elected by a majority vote.
		\item All remaining positions are filled by the candidate reaching most votes from each Committee, which isn't already chosen as a Member of Parliament in \textbf{\S1}. If there is no such candidate, the candidate with most votes (from any Committee) that isn't already chosen will be chosen as Member of Parliament.
	\end{parenum}



	\protect\section{Election of the College and Focus area representatives}
	Any eligible student according to section \ref{eligible} may run as a college or Focus area representative in their College or Focus area. Focus area and College representatives are elected by the USB from within their Focus area or College.



	\protect\section{Election of the President's office}
	\label{PresiElection}
	After the election in May, the newly elected Parliament shall elect a President and after the election in December, they shall elect a Vice-President, each amongst the Members of Parliament. Both President and Vice-President can be recalled any time with an absolute 2/3 majority vote in the Parliament. In this case, a new President or Vice-President must be elected immediately.


\section{Term Length}
By default, all elected positions' terms within the USG shall be one year long.
\begin{parenum}
	\item Exceptions are:
	\begin{enumerate}
		\item 
		the candidates are third year students elected in the December election of their third year, and they are graduating at the end of their 6\textsuperscript{th} semester,
		\item 
		the candidates are second year students elected in the December election of their second year, and they are going abroad during their 5\textsuperscript{th} semester,
		\item
		the candidates are running for an open position from a previous one-semester term,
		
		\item elected members choose to have a one semester term.
	\end{enumerate}
	\item The current members at the time of the election will remain in office until the end of the respective semester. After the elections, the newly elected members are already invited as standing guests in the meetings. They obtain voting power already at the beginning of the respective break (winter or summer break). In particular, this means that for the breaks between semesters, both previous and newly-elect students are members of the Parliament and thus have voting rights.
	\item A student can run for a position as many times as they wish.
	
\end{parenum}
