\article{Campus Affairs Committee By-Laws}
\label{CACByLawsdef}

\sectionstar{Duties of the Campus Affairs Committee}
\begin{parenum}
\item The primary purpose of the Campus Affairs Committee (CAC)  is to facilitate and improve life on campus for students. The CAC shall consider and discuss with the appropriate university authorities any relevant issue for which there is a proven student interest, as long as the area of interest does not infringe upon the areas supervised by other Committees, determined by either of:
\begin{enumerate}
\item the Committee and/or the Parliament's discretion,
\item any student petition that has more than 50 signatures.
\end{enumerate}

\item Any student shall have the right to approach the CAC for initiation of the proceedings described above upon consideration of the \hyperref[USGParliamentDef]{USG Parliament}.
The CAC shall have the right to file an official complaint to Campus Life against any employee on campus provided that there exists evidence beyond reasonable doubt on the issue, and with an absolute majority vote of the Parliament.

\item \add[N]{more text}
% TODO: write it\end{parenum}
\end{parenum}



\add[P]{\protect\section{Tasks}
The Committee shall work on relevant suggestions and requests from individual students or other institutions. }\add[S]{The fixed and recurrent tasks of the committee shall pertain to:}\add[P]{
\protect\begin{enumerate}
    \item delegating an USR (usually the CAC Chair) to attend the weekly meetings with CampusLife,
    \item maintaining the weekly CAC meeting as mutually agreed by the entire CAC,
    \item evaluate campus life policies give feedback about non-academic matters to CampusLife,
    \item be responsible for meeting university leadership at least once a semester to discuss non-academic developments,
    \item regularly update the \protect\hyperref[studentbody]{USB} and the Parliament about its workings as well as about meetings,
    \item oversee and foster the student clubs and events, as well as doing campaign about social welfare.
\end{enumerate}}

\add[S]{\protect\section{Sub-Committees}
Current sub-Committees are, but not restricted to:
\protect\begin{enumerate}
    \item Food and Environmental SubCommittee
    \protect\begin{enumerate}
        \item ursuing students' interest in all things related to University-provided food,
        \item Striving to restore and maintain the environmental integrity of the campus,
        \item Creating various projects and workgroups to pursue the above interests.
    \end{enumerate}
    
    \item Equality and Diversity SubCommittee:
    \protect\begin{enumerate}
        \item Headed by a current Undergraduate Student Representative of the Equality and Diversity Committee (EQC), 
        \item Working closely with the EQC and assisting it with its current projects wherever possible, 
        \item Working with other clubs and communities on campus that aim for a similar goal, 
        \item Creating various projects and workgroups to purse the above interests. 
    \end{enumerate}
    
    \item Event SubCommittee
    \protect\begin{enumerate}
        \item Assisting student organisers in planning for various events and parties,
        \item Organising the end-of-the-year party. 
    \end{enumerate}
\end{enumerate}}


\section{Student Clubs}
Any student shall have the right to ask the \hyperref[USGParliamentDef]{USG Parliament} to check the respective club`s inner workings. \textit{Only for clubs that receive USG funding,} any club member shall have the right to ask the USG to impeach the respective club`s president and hold club elections. 
\begin{parenum}
\item Should an impeachment proceeding occur, the decision shall be decided by a simple majority vote of the USG Parliamentarians.
\item Should a member of the \hyperref[USGdef]{USG} be involved in the impeachment procedure (accused or member of the club), he/she shall not have the right to vote on the impeachment for the sake of impartiality.
\end{parenum}

\add[P]{\protect\section{Composition and Roles}
The CAC shall be represented by students of all residences and student clubs. Each College shall elect its representative, who is a member of the Parliament as well. Students may become members of the CAC representing their own communities and/or student clubs.

Special roles within the CAC are, but not limited to:
\protect\begin{enumerate}
    \item CAC Chair
    \item Two Undergraduate Student Representatives in the Equality and Diversity Committee
    \item Campus Life Ombudsperson
    \item CAC Secretary
\end{enumerate}}


\add[S]{
	\protect\section{College representatives}
	The elected College representatives shall become CAC members. The College area representatives shall:
	\begin{enumerate}
		\item serve as a point of contact for students of their college,
		\item work on issues specific to their college,
		\item represent the interest of their College in CAC internal discussions, thus ensuring the needs of all colleges are taken into account,
		\item stay in contact with and support the work of the college office and organizers of college specific events in their college, serving as a link between the College offices and event organizers and the CAC.
	\end{enumerate}
}