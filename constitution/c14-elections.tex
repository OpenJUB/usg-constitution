\article{Election Procedures}
\label{electionprocedurebylaws}

The provisions of this article shall only apply to elections for positions in the USG.

\section{Election Committee} 
The election Committee (EC) consists of the elected members of the USG who do not finish their term at the respective election time. If requested, they shall be supported by any remaining members. The election committee has to form itself at the president`s initiative. The committee then elects a head of elections. 

\section{Special Elections} 
Any other elections (except the two specified by the constitution in \ref{electiontimes}) during the academic year shall be organized by the President of the USG who chairs the respective election Committee.

\section{Conflict of Interest}
Each Parliament and secretariat member shall have the duty to report any irregularities to \hyperref[StudentCourtDef]{Student Court}. No member of the student court, Parliament and secretariat who is supervising the election shall be a candidate in or initiator of the election in question.

\section{Rules of Procedure} 
The election is carried out in accordance with the provisions set forth in the Constitution of the \hyperref[studentbody]{USB} and these provisions.

\section{Candidates' Applications}
Students wishing to apply shall submit their application to the EC, which will collect the applications and prepare them for the election. Applications may consist of a letter of intent (one-pager), explaining the motivation for applying, past experience in student governments or university administration, an outline of an agenda they plan to pursue and proposed solutions, a strategy for engaging with the USG and other stakehoders.


\section{Election Announcements} 
All elections shall be announced no later than two (2) days prior to the election, naming the options or candidates and functions they are running for, place, date, and time of the election. Such announcements shall take place through a notification sent to all student email accounts and by other platforms set by the USG via the EC.

\remove[S]{\protect\section{Candidates' Eligibility}
The student shall be physically present in the respective semester of being in the office.
\protect\begin{parenum}
\item
Students in their third year of study may not run for an office in the USG in the May elections of their last year.

\item
Students in their second year of study who will spend the fifth semester abroad or will do an internship in their fifth semester may not run for an office in the USG in the May elections of their second year.

\item
Foundation Year students may only run for an office if they have been offered admission for undergraduate studies at Jacobs University and have accepted that offer.

\item 
Only in exceptional cases, as decided by the \hyperref[USGParliamentDef]{USG Parliament}, one student is allowed to be the Undergraduate Student Representatives \add[S]{(excluding substitutes) }of more than one official Committees of Jacobs University with student representatives (USR).\note[N]{\\Maybe restrict chairs from being members of too many committees instead. Given the high number of USRs and substitutes, this might otherwise bee too restrictive.}

\end{parenum}}

\section{Announcing the Candidates}
The EC is responsible for officially announcing the candidates.
\begin{parenum}
\item
Candidates in or initiators of an election shall be allowed to have an official public profile using an online forum set-up by the EC, in order to present their ideas and opinions relevant to the election.
\item
A General Assembly (GA) organized by the EC will take place during the campaigning period, i.e. after the candidates announcement and before the voting starts.
\item
The purpose of the GA is to (physically) present the candidates, and to have a debate forum where any member of the Jacobs Community shall have the opportunity to ask questions, especially towards the candidates.
\end{parenum}

\section{Campaigning}
Candidates shall have the right to democratically campaign in an appropriate and honest way. The Election Committee shall have the responsibility to more explicitly define the above, if requested by any student. All candidates agree and accept that:
\begin{enumerate}[nosep] 
\item
campaigning is only allowed during the campaign period,
\item
slander is strictly prohibited,
\item
in the case that candidates by their own wish shall need financial resources for campaigning, no funding from USG or Campus Life shall be used.
\end{enumerate}


\section{Candidates' Misconduct} 
The election Committee shall have the right to establish additional regulations regarding public campaigning if deemed appropriate and impose sanctions/disqualify candidates. These additional regulations must comply with the Constitution of the \hyperref[studentbody]{USB}.

%\section{Time and Place of Elections} 
%All elections shall take place in the college serveries or a similar adequate room that is to be determined by the election Committee, during lunch and dinner on the day(s) of the election.

\remove[S]{\protect\section{Voter Eligibility} 
Every member of the \protect\hyperref[studentbody]{USB} who has not lost his or her right to vote shall be eligible to vote in any election by the \protect\hyperref[studentbody]{USB}.} 

\section{Online Ballots} 
The IT Department has the task to set up an election platform for online voting. Ballots have to be approved by the election Committee prior to the election. To be considered fit for being used in voting, a ballot has to contain all options presented in an equal way without discriminating any of them.

\section{Violations} 
Violations of the election rules and regulations set forth in these By-laws may result in:
\begin{enumerate}[nosep]
\item the violator losing his or her status as a candidate in the election
\protect\remove[S]{\protect\item the violator losing his or her status as eligible voter for future elections}\protect\note[Colin]{wtf?}
\item the election being invalid
\item new election results being issued.
\end{enumerate}
Such cases shall be handled and decided upon by the EC.

\section{Publication of results}
Immediately after the vote-tallying is finished, the election Committee shall inform the secretariat about the results of the election. The secretariat shall then publish the results of the election, including the number of eligible voters, the number of votes cast, the number of valid votes, the number of invalid votes, the number of votes for each of the options or candidates, the number of votes for each of the options or candidates per college, and naming the winning option or candidate. This shall be done through a notification sent to all student email accounts and an announcement posted on the student forum.

\section{Taking Office}
Once the newly-elected members of the \hyperref[USGParliamentDef]{USG Parliament} are determined, the newly-elect members of the \hyperref[USGParliamentDef]{USG Parliament} and the outgoing members take responsibility that the newly-elected members of the \hyperref[USGParliamentDef]{USG Parliament} are trained in their respective tasks.