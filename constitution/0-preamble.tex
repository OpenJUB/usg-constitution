\add[S]{Throughout the document the following markups are used to denote changes:\\}
\add[P]{Parliament approved additions.\\}
\add[S]{Suggested additions.\\}
\note[N]{Notes.}
\remove[RP]{\\Parliament approved removals.\\}
\remove[RS]{Suggested removals.\\}
\change[S]{original}{suggested replacements}
\begin{center}
\textbf{Preamble}
\end{center}

{\parskip=\baselineskip
Acting as the voice of all students enrolled at Jacobs University, the \acf{USG} shall actively work to improve the quality of life (in all its aspects) for the \acf{USB}, along with the university administration and community as a whole. The purpose of the \acf{USG} is to work in the best interest of the student body.

According to the \href{https://usg.jacobs.university/wp-content/uploads/2018/02/Academic_Constitution_2017_english.pdf}{Jacobs University Academic Constitution}, "the responsibilities of the Student Government include representing \textit{Students} towards the respective bodies within as well as outside Jacobs University, serving as the link between Students and university authorities, administration or other groups on campus and actively contributing to the communication between those bodies, appointing all representatives of students, and ensuring continuity."


Believing in the right of self-governance, the \acf{USB} shall elect its representatives. Its \acf{USG} is entirely student-run and politically independent of any administrative bodies. 
}