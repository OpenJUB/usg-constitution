\article{Definitions}

\section{Constitution}
Unless stated otherwise, the term "constitution" refers to the text of the present constitution, as legislative document of the Undergraduate Student Body (USB).
 
\section{Students and Undergraduate Student body (USB)}
\label{studentbody}
 Unless stated otherwise, the term "students" refers to undergraduate students as defined in the university constitution, i.e. Bachelor and pre-degree students currently enrolled at Jacobs University. Unless stated otherwise, the term "student body" refers to the Undergraduate Student Assembly, as defined by Jacobs University constitution.

\section{\add[P]{Member of Parliament}}
\label{MPDef}
\change[S]{\protect\add[P]{The Members of the \protect\hyperref[USGParliamentDef]{USG Parliament}, Members of Parliament or MP for short, are the President, Vice-President, Treasurer, College Representatives and Focus Area Representatives.}}{The Members of the \protect\hyperref[USGParliamentDef]{USG Parliament} of MP for short, are USG committee members elected into the Parliament as described in section \protect\ref{MPElectionDef}.}

\section{\add[P]{Member of the USG}}
\label{MemberUSGDef}
\add[P]{All members of the \protect\hyperref[USGParliamentDef]{USG Parliament}, all members of the \protect\hyperref[USGexecutiveUnitDef]{USG Executive Unit}}}\add[P]{, all Secretaries or members of Support Structures }\add[S]{and all major-representatives }are Members of the USG.

\section{Undergraduate Student Representative (USR)}
\label{USRdef}
The Undergraduate Student Representative (USR) is the person physically participating at the University meeting, as stipulated by the Jacobs University Constitution.

\section{Provision}
A provision is a complementary law adopted by the Student Government according the constitution. They shall be referenced in the text of the constitution as soon as they are adopted. They are approved or amended by a simple majority vote of the \hyperref[USGParliamentDef]{\textcolor{red}{USG Parliament}}.

\section{\add[P]{Rules of Procedure}}
\add[P]{Rules of Procedure are complementary provisions of a unit, a Committee or the election committee regarding their internal working procedures. Consequently, they don't violate nor affect the USG constitution and existent bylaws which are not themselves Rules of Procedure. They also don't affect any other constituency of the USG. \protect\change[S]{Thus, t}{T}hey shall be amended by a simple majority of members of the respective unit or committee and approved by a simple majority of the} \protect\hyperref[USGParliamentDef]{USG Parliament}.

\section{Student Club}
A student club is a student-run organization focused on a social/academic/sport or leisure purpose. To be official, it has to be registered with Campus Life.

\section{Student Budget}
The student budget refers to the aggregate amount of money paid by students as the USG tax collected by the university.

\section{Popular Elections}
All elections in the \hyperref[studentbody]{USB} shall be general, direct, free, fair, secret and held accordingly to the Election Procedure By-Laws of Article \ref{electionprocedurebylaws}.

\section{Referendum}
\label{referendum}
A referendum is a legislative act of the USG decided upon by a popular vote of the \hyperref[studentbody]{USB}. The outcome of a referendum shall be binding on the USG.
\begin{parenum} 
\item
The organization of a referendum requires a simple majority of all Members of the USG to pass it to a student vote. 
\item
A referendum is considered to be successful if a simple majority votes in favour, with a threshold quorum of a fifth (20\%) of all eligible voters.
\item
No decisions shall be implemented if a referendum is pending on an issue.
\end{parenum}


\section{Student Opinion}
\label{StudentOpiniondef}
The USG may consult the \hyperref[studentbody]{USB} for future decisions via the Student Opinion on the \href{https://vote.jacobs.university/}{voting platform}. Its purpose is to give more information and leverage to the actions of the USG. However, the outcome of the Student Opinion shall be binding on the USG, unless the \hyperref[USGParliamentDef]{USG Parliament} votes in absolute unanimity against it.

\section{Popular Initiative} 
 A popular initiative is an issue, not voted on by the USG, relating to the \hyperref[studentbody]{USB}. It is put directly to a popular vote by the \hyperref[studentbody]{USB} without prior consultation with the USG. The outcome of the popular initiative shall be binding on the USG. The organization of popular initiatives requires a petition signed by at least twenty (5\%)  of all students. The Members of the USG shall approve the initiative or add options to the ballot to ensure feasibility with an absolute majority vote. Ballot options included on the petition for the popular initiative may only be excluded by the Members of the USG for reasons of feasibility.

\section{\add[P]{General Assembly of the USB}} 
\add[S]{A general assembly is an official meeting called for by the USG to which all undergraduate students are invited.}

\section{Absolute and Simple Majority}
An absolute majority refers to a voting outcome in which more than fifty percent (50\%) of the total number of the respective constituency (present or not) votes in favor.  A simple majority refers to a voting outcome in which there are more "in favor" votes than "against" votes among the physically present members\add[S]{; abstentions are part of the pool.}

\section{\remove[S]{Absolute and }Simple Unanimity}
\remove[S]{An absolute unanimity refers to a voting outcome in which all members of the respective constituency (present or not) votes in favor. }A simple unanimity refers to a voting outcome in which all physically present members vote in favour.\add[S]{ Additionally, a majority of members of the respective constituency needs to be present during the vote.}


\section{Campaign Period}
The period from announcing a candidature and before the voting for it starts shall be known as the campaign period.

\section{Impeachment}
The process of removing someone from their office.