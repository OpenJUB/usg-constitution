\article{USG Parliament}
\label{USGParliamentDef}
The USG Parliament shall have the role to coordinate the USG, to explicitly define its tasks and to distribute workload. The \hyperref[USGexecutiveUnitDef]{USG Executive Unit} shall fulfill the proposals of the USG Parliament, upon common agreement between the two bodies.

\section{Structure and Elected Offices} 
\label{ElectedOfficesDef}
The Undergraduate Parliament shall be composed of \add[S]{ten (10) elected Members of Parliament:}\remove[RS]{the Elected Offices:  
\protect\begin{multicols}{2}
\protect\begin{enumerate}
\item President
\item \protect\label{CRepDef}College Representative: Krupp 
\item College Representative: Mercator 
\item College Representative: College 3
\item College Representative: Nordmetall
\item Vice-President
\item \protect\label{FARepDef} Focus Area Representative: Diversity
\item Focus Area Representative: Health
\item Focus Area Representative: Mobility
\item \protect\label{TreasurerDef} Treasurer
\end{enumerate}
\end{multicols}}
\add[S]{In order to ensure fair representation of different interest groups on campus, there are additionally the following elected offices:
\protect\begin{multicols}{2}
	\begin{enumerate}
		\item \label{CRepDef}College Representative: Krupp 
		\item College Representative: Mercator 
		\item College Representative: College 3
		\item College Representative: Nordmetall
		\item \label{FARepDef} Focus Area Representative: Diversity
		\item Focus Area Representative: Health
		\item Focus Area Representative: Mobility
	\end{enumerate}
\end{multicols}The college representatives become members of the Campus Affairs Committee and the Focus area representatives become member of the Academic Affairs Committee, as detailed in the respective bylaws.}

\section{Elections}
\label{electiontimes}
The USG shall hold a general election twice a year, in particular:
\begin{enumerate}

\item in first week of December for: \remove[RS]{Vice-President, }Focus Area representatives, Treasurer\add[S]{, 5 Members of Parliament } (and for any open position),
\item in first week of May for: \remove[RS]{President, }College representatives\add[S]{, 5 Members of Parliament} (and for any open position).
\end{enumerate}

\add[S]{
\protect\section{Election of the Members of Parliament}
\add[S]{Only active members of an USG Executive Committee are eligible to run as a Member of Parliament. }
For the election of the Member of Parliament, each student may distribute three (3) votes on candidates. Then, the Members of Parliament are selected based on the outcome of the election as follows:  
\protect\begin{parenum}
	\item Firstly, the candidates with most votes (except for a number of candidates equal to the number of executive Committees) become Member of Parliament. 
	%Firstly, the $5-n$ candidates with most votes (where $n$ denotes the number of executive Committees of the USG) become Member of Parliament. For each Parliament position to be refilled (due ot resignations of impeachments) one additional candidate is elected by a majority vote.
	\item All remaining positions are filled by the candidate reaching most votes from every Committee, which isn't already chosen as a Member of Parliament in \textbf{\S1}.
\end{parenum}
}

\add[S]{
\protect\section{Election of the President's office}
\label{PresiElection}
In the beginning of each Fall semester, the newly elected Parliament shall elect a President and in the beginning of each Spring semester they shall elect a Vice-President amongst the Members of Parliament. Both President and Vice-President can be recalled any time with an absolute 2/3 majority in the Parliament. In this case, a new President must be elected immediately.
}

\section{Term Length}
By default, all elected positions' terms within the USG shall be one year long.
\begin{parenum}
\item Exceptions of the above, with one-semester terms, are:
\begin{enumerate}
\item 
the candidates are third year students elected in the December election of their third year, and they are graduating at the end of their 6\textsuperscript{th} semester,
\item 
the candidates are second year students elected in the December election of their second year, and they are going abroad during their 5\textsuperscript{th} semester,
\item
the candidates are running for an open position from a previous one-semester term,
\item
the candidates receive the second highest number of votes in an election where two positions are to be filled,
\item
the candidates receive the second highest number of votes in an election where any Member of Parliament runs and is elected for a different position within the Parliament,
\item
elected members choose to have a one semester term.
\end{enumerate}
\item
The current members at the time of the election will remain in office until the end of the respective semester. After the elections, the newly elected members are already invited as standing guests in the meetings. They obtain voting power already at the beginning of the respective break (winter or summer break). In particular, this means that for the breaks between semesters, both previous and newly-elect students are members of the Parliament and thus have voting rights.
\item
A student can run for a position as many times as they wish.

\end{parenum}