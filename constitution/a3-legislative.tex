\article{USG Parliament}
\label{USGParliamentDef}
The USG Parliament shall have the role to coordinate the USG, to explicitly define its tasks and to distribute workload. The \hyperref[USGexecutiveUnitDef]{USG Executive Unit} shall fulfill the proposals of the USG Parliament, upon common agreement between the two bodies.

\section{Structure and Elected Offices} 
\label{ElectedOfficesDef}
The Undergraduate Parliament shall be composed of the Elected Offices:  
\begin{multicols}{2}
\begin{enumerate}
\item President
\item \label{CRepDef}College Representative: Krupp 
\item College Representative: Mercator 
\item College Representative: College 3
\item College Representative: Nordmetall
\item Vice-President
\item \label{FARepDef} Focus Area Representative: Diversity
\item Focus Area Representative: Health
\item Focus Area Representative: Mobility
\item \label{TreasurerDef} Treasurer
\end{enumerate}
\end{multicols}

\section{Elections}
\label{electiontimes}
The USG shall hold a general election twice a year, in particular:
\begin{enumerate}

\item in first week of December for: Vice-President, Focus Area representatives, Treasurer (and for any open position),
\item in first week of May for: President, College representatives (and for any open position).
\end{enumerate}

\section{Term Length}
By default, all elected positions' terms within the USG shall be one year long.
\begin{parenum}
\item Exceptions of the above, with one-semester terms, are:
\begin{enumerate}
\item 
the candidates are third year students elected in the December election of their third year, and they are graduating at the end of their 6\textsuperscript{th} semester,
\item 
the candidates are second year students elected in the December election of their second year, and they are going abroad during their 5\textsuperscript{th} semester,
\item
the candidates are running for an open position from a previous one-semester term,
\item
the candidates receive the second highest number of votes in an election where two positions are to be filled,
\item
the candidates receive the second highest number of votes in an election where any Member of Parliament runs and is elected for a different position within the Parliament,
\item
elected members choose to have a one semester term.
\end{enumerate}
\item
The current members at the time of the election will remain in office until the end of the respective semester. After the elections, the newly elected members are already invited as standing guests in the meetings. They obtain voting power already at the beginning of the respective break (winter or summer break). In particular, this means that for the breaks between semesters, both previous and newly-elect students are members of the Parliament and thus have voting rights.
\item
A student can run for a position as many times as they wish.

\end{parenum}