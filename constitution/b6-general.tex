\article{General Provisions}
\label{generalProvisions}

\section{Decision-Making and By-Laws} 
A simple majority among the committee members is required to reach a decision within any committee. \add[S]{Conflicts of interest must be clearly declared.}

%CONFLICT OF INTEREST MUST BE CLEARLY DECLARED

\add[P]{The committees decision can be overruled only with a full unanimity of the \protect\acl{PARL}. Each committee shall establish its own By-Laws, afterwards to be approved by the Parliament.}

\section{Membership}
\add[P]{Next to the Committee chair, the Committee consists of regular members, which are inaugurated by vote of the committee after an application process to the respective Committee chair. Memberships continue until disenrollment, unless the member resigns or is inactive (as decided by the \protect\add[P]{respective committee}). Members are either:}
\begin{enumerate}
    \item active members \add[P]{with voting rights within the Committee and the USG, who need to attend the mandatory Committee meetings}
    \item passive members \add[P]{as part of the mailing lists, with the option to join the meetings and engage in discussions, but without voting rights}.
\end{enumerate}

\add[S]{\protect\section{Committee Responsibilities}
\protect\begin{enumerate}
    \item Maintaining a detailed documentation of the projects in progress and/or finished,
    \item Creating a report to be presented to the USG Parliament during its weekly meetings.
\end{enumerate}}
    
\add[S]{\protect\section{General workings of the subcommittees}
Subcommittees can be created within the Committee in order to foster the work with the diverse range of issues the Committee is interested in. The current tasks and responsibilities are: 
\protect\begin{enumerate}
    \item Membership of any of these subcommittees is fluid and can be freely changed.
    \item A committee member can freely decide to be in any number of such subcommittees.
    \item The subcommittee is free to accept any non-committee member into its membership.
    \item The heads of the subcommittees are decided internally by the interested members
\end{enumerate}}

\section{\add[P]{The Chair}\add[S]{of a committe}}
The Chair usually is an elected member of the Parliament. \add[P]{However, the committee may elect its own chair among its own members, if deemed necessary.}\add[S]{ In this case, the chair will be invited as a standing guest to the USG Parliament and given voting rights for issues concerning the responsibility of their committee.} The chair takes full accountability and responsibility for the work of and within the committee and, thus, it shall be the duty of the chair to ensure that all committee tasks are being duly implemented. Each committee defines additional duties.

\section{\add[P]{Powers and Duties of the Chair}}
The Chair is a leading figure for their committee. They shall ensure the following.
\begin{parenum}
\item Communication with the USG Parliament. The chair shall \add[P]{attend the USG Parliament meetings}, represent and communicate relevant projects and suggestions of the committee work to the USG Parliament and vice versa. 

\item The chairs shall preside over the meetings of their committee.

\item At the beginning of each year, the chair must present a project list for the respective task and how to accomplish them. The chair shall write committee reports if required by the \hyperref[USGdef]{USG}.

\item Establishment of the committee. Upon assignment of the committee the chair shall nominate committee members. 

\item It is at the discretion of the chair to invite additional members to attend their respective official meetings. If unable to attend, the chair is responsible for finding substitutes.

\item The chair shall provide general advice concerning the committee's work for the succeeding committee chair, specifying the activities which were pursued and completed, stating any pending activities, and including suggestions for improved performance in the committee in the next academic year. The chair shall ensure that all relevant documents and materials are handed over to the succeeding committee chair. 
\end{parenum}

\section{Impeachment}
In case of a failure of any committee member (including the chair) to live up to his/her responsibilities with regards to the committee, the USG Parliament shall take appropriate actions, ranging from a reprimand to impeachment and/or a re-delegation of tasks.

\add[P]{
After a successful impeachment of a committee member, an additional member shall be elected or appointed as soon as possible to fill the open position.}

\add[P]{\protect\section{Change of Responsibilities}
Any changes related to responsibilities, tasks and duties of the committee as well as procedures and guidelines within the committee that are of a permanent nature must be approved by the USG Parliament by a simple majority, and those of temporary nature are at the discretion of the chair.}