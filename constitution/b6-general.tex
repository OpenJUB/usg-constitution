\article{General Provisions\add[S]{for the Executive Committees}}
\label{generalProvisions}

\section{Membership}
\add[P]{Next to the Committee chair, the Committee consists of regular members, which are inaugurated by a vote of the Committee\protect\remove[S]{ after an application process to the respective Committee chair}.\note[Colin]{Should be left to the Committee to decide} Memberships continue until disenrollment, unless the member resigns or is inactive (as decided by the respective committee). Members are either:}
\begin{enumerate}
	\item active members \add[P]{with voting rights within the Committee and the USG, who need to attend the mandatory Committee meetings}
	\item passive members \add[P]{as part of the mailing lists, with the option to join the meetings and engage in discussions, but without voting rights}.
\end{enumerate}


\section{Decision-Making and By-Laws} 
A simple majority among the Committee members is required to reach a decision within any committee. \add[S]{Conflicts of interest must be clearly declared.}

%CONFLICT OF INTEREST MUST BE CLEARLY DECLARED

\add[P]{The Committee's decision can be overruled only with a \protect\change[S]{full}{simple} unanimity of the \protect\acl{PARL}. Each committee shall establish its own By-Laws, afterwards to be approved by the Parliament.}

\section{\add[P]{The Chair}\add[S]{ of a Committee}}
\change[S]{The Chair usually is an elected member of the Parliament.}{Each semester after the USG elections each committee shall elect its chair amongst its members. The chair can, but doesn't need to be a Member of Parliament. In case the chair is not a Member of Parliament, they will be invited as a standing guest to the Parliament to represent their committee. They are given the right to propose agenda points (including ad-hoc agenda points and comments) and they will be included in the Parliament internal communications, however they don't have voting rights.} The chair takes full accountability and responsibility for the work of and within the committee and, thus, it shall be the duty of the chair to ensure that all committee tasks are being duly implemented. Each committee \change[S]{defines}{may define} additional duties.

\section{\add[P]{Powers and Duties of the Chair}}
The Chair is a leading figure for their Committee. They shall ensure the following.
\begin{parenum}
\item Communication with the USG Parliament. The chair shall \add[P]{attend the USG Parliament meetings}, represent and communicate relevant projects and suggestions of the Committee work to the USG Parliament and vice versa. 

\item The chairs shall preside over the meetings of their Committee.
\item At the beginning of each year, the chair must present a project list for the respective task and how to accomplish them. The chair shall write Committee reports if required by the \hyperref[USGdef]{USG}.

\item Establishment of the Committee. Upon assignment of the committee the chair shall nominate committee members. 

\item It is at the discretion of the chair to invite additional members to attend their respective official meetings. If unable to attend, the chair is responsible for finding substitutes.

\item The chair shall provide general advice concerning the Committee's work for the succeeding committee chair, specifying the activities which were pursued and completed, stating any pending activities, and including suggestions for improved performance in the committee in the next academic year. The chair shall ensure that all relevant documents and materials are handed over to the succeeding committee chair. 
\item At any time during the semester the chair of a Committee, may be recalled by an absolute 2/3 majority of members. In this case, a new chair must be elected immediately.
\end{parenum}

\add[S]{
\protect\section{Moderator of a Committee}
The chair may yield their moderation right to another Committee member, either for a specific meeting or for the entire semester. The moderator may yield the moderation right to a different committee member at any time during a meeting for:\protect\begin{enumerate}
		\item the discussion of an agenda point,
		\item the introduction of a proposal,
		\item the remaining meeting.
\end{enumerate}}

\add[S]{\protect\section{General workings of the subcommittees}
	Subcommittees can be created within the Committee in order to foster the work with the diverse range of issues the Committee is interested in.} \note[N]{Deleted the suggested phrase: The current tasks and responsibilities are: }\add[S]{
	\protect\begin{parenum}
		\item Membership of any of these subcommittees is fluid and can be freely changed.
		\item A Committee member can freely decide to be in any number of such subcommittees.
		\item The subcommittee is free to accept any non-committee member \change[S]{into its membership}{as members}.
		\item The heads of the subcommittees are \change[S]{decided internally by the interested members}{selected within the subcommittee}
\end{parenum}}

\remove[RS]{\protect\section{Impeachment}
In case of a failure of any Committee member (including the chair) to live up to his/her responsibilities with regards to the committee, the USG Parliament shall take appropriate actions, ranging from a reprimand to impeachment and/or a re-delegation of tasks.

\add[P]{After a successful impeachment of a Committee member, an additional member shall be elected or appointed as soon as possible to fill the open position.}

\protect\add[P]{\protect\section{Change of Responsibilities}
Any changes related to responsibilities, tasks and duties of the Committee as well as procedures and guidelines within the committee that are of a permanent nature must be approved by the USG Parliament by a simple majority, and those of temporary nature are at the discretion of the chair.}}