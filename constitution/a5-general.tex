\add[S]{
\protect\article{General}

\add[S]{\protect\section{Impeachment}
	In case of a failure of any Committee member (including the chair) to live up to his/her responsibilities with regards to the committee, the USG Parliament shall take appropriate actions, ranging from a reprimand to impeachment and/or a re-delegation of tasks.
	
	\add[P]{
		After a successful impeachment of a Committee member, an additional member shall be elected or appointed as soon as possible to fill the open position.}}

	\add[P]{\protect\section{Change of Responsibilities}
		Any changes related to responsibilities, tasks and duties of the Committee as well as procedures and guidelines within the committee that are of a permanent nature must be approved by the USG Parliament by a simple majority, and those of temporary nature are at the discretion of the chair.}

	\protect\section{External Outreach Power}
	The USG shall have the right to communicate with external parties, as stated by Academic Constitution of Jacobs University Bremen. 
	\protect\begin{parenum}
		\item The USG members, as any student, shall have the right to contact any external bodies, only if they do so as individuals and not representing the USG.
		\item The whole USG (Parliament and Executive Unit) has to be informed of any intention for external communication.
		\item A simple majority vote in the \hyperref[USGParliamentDef]{USG Parliament}, among the USG members or a Student Opinion Decision may enact or prevent any external communication.
		\item In particular, the USG President is the main responsible for any external communication.
	\end{parenum}
	
	\protect\section{Communication Responsibilities}
	One of the USG`s core missions is to inform the \hyperref[studentbody]{USB}. 
	\protect\begin{parenum}
		\item A regular task shall be a monthly newsletter of one page with USG past, current and future projects relevant to the respective time-frame.
		\item Failure to do so for two months consecutively constitutes a valid reason for impeachment.
	\end{parenum}
	
	\protect\section{Conflicts of Interest}
	Conflicts of interest must be declared and the involved members must be omitted from voting. In case of a tie in the voting procedure, the final decision is made jointly by the President, Vice-President and the respective Committee Head.
	
	
	
	\protect\section{Resignation and Impeachment}
	In the event of an USG Member resigning, not being able, not being willing to continue their tasks as a Member, or being successfully impeached, the \hyperref[USGParliamentDef]{USG Parliament} will appoint an interim for the remaining period of the resigning member.
	
	\protect\begin{parenum}
		\item Any Member of the USG may resign provided that the USG is informed two (2) weeks before the intended date of resignation.
		
		\item Any member of the USG shall be impeached by either of the two: Two thirds (2/3) of the (other) elected members or referendum. Such a vote or referendum must clearly state the grounds for impeachment. Repeated \add[S]{unexcused }absence \add[P]{of three (3) }\remove[RS]{consecutive}\add[P]{ meetings or more} and/or not accomplishing designated tasks automatically start the process of impeachment of a member:
		\begin{enumerate}
			\item The accused member and the impeacher (if any) have one (1) week time to argument and present their case.
			\item At the next \hyperref[USGParliamentDef]{USG Parliament} meeting, the impeachment is voted upon.
		\end{enumerate}
		
		\item After resignation or impeachment, an officer may not run again for the same office within the term that the resignation \add[P]{or impeachment took place}.
	\end{parenum}




}