\article{General Regulations}
	\section{Decision-Making and By-laws} 
	A simple majority among the Committee members is required to reach a decision within any committee. Conflicts of interest must be declared and the involved members must be omitted from voting. 

%CONFLICT OF INTEREST MUST BE CLEARLY DECLARED

	The Committee's decision can be overruled only with a simple unanimity of the \acl{PARL}. Each committee shall establish its own By-laws, afterwards to be approved by the Parliament.
	\protect\section{Committee Responsibilities}
		Each Committee shall:
		\protect\begin{enumerate}
			\item maintain a detailed documentation of the projects in progress and/or finished,
			\item update the USG Parliament about its work during its weekly meetings.
			\protect\protect\item work on any tasks assigned to them by the USG Parliament and keep the Parliament updated about any progress
	\end{enumerate}

\protect\section{Change of Committee Responsibilities}
		Any changes related to responsibilities, tasks and duties of an Executive Committee as well as procedures and guidelines within that committee that are of a permanent nature must be approved by a simple majority vote of the USG Parliament, and those of temporary nature are at the discretion of the Committee.

	
	\protect\section{Communication Responsibilities}
	One of the USG`s core missions is to inform the \hyperref[studentbody]{USB}. 
	\protect\begin{parenum}
		\item A regular task shall be a bi-monthly newsletter with USG past, current and future projects relevant to the respective time-frame.
		\item Failure to do so for two months consecutively constitutes a valid reason for impeachment of the USG President.
	\end{parenum}

	\section{Support structures} 
	\label{suppstrucdef}
	The Support structures shall foster the inner workings of the USG and help both the \hyperref[USGParliamentDef]{USG Parliament} and Executive Unit in their actions. Further roles can be defined and must be publicly justified, then they can be attributed to USG members depending on the respective need, in addition to the support structures explicitly listed below:
	\begin{parenum}
		\item The Secretaries shall:
		\begin{enumerate}
			\item set the Agenda for meetings of USG bodies,
			\item be responsible of communication and documents, such as agendas and minutes,
			\item organize and follow up on tasks, check for completion;
			\item organize and update the USG documentation, ensuring full transparency of the USG's actions
		\end{enumerate}
		
		\item The IT Department shall:
		\begin{enumerate}
			\item be responsible for the maintenance of the USG related websites and for the updated information,
			\item be the link between the \hyperref[studentbody]{USB} and the IT department of the University.
		\end{enumerate}
		
		\item Public Relations Department shall:
		\begin{enumerate}
			\item be responsible for the image of the USG, and implicitly of the University, 
			\item cooperate with the Parliament on internal and external communication.
		\end{enumerate}
	\end{parenum}
	These roles shall be positions any student can apply for, then the candidate is selected within the USG Parliament or the respective Committee.
	
	\protect\section{External Outreach Power}
	The USG shall have the right to communicate with external parties, as stated by Academic Constitution of Jacobs University Bremen. 
	\protect\begin{parenum}
		\item The USG members, as any student, shall have the right to contact any external bodies, only if they do so as individuals and not representing the USG.
		
		\item In particular, the USG President is the main representative responsible for any external communication.
	\end{parenum}
		
	\protect\section{Impeachment of USG members}
		In case of a failure of any USG member to live up to his/her responsibilities with regards to the committee, the USG Parliament shall take appropriate actions, ranging from a reprimand to impeachment and/or a re-delegation of tasks.
	
	\protect\section{Resignation and Impeachment Procedure}
	In the event of an elected USG Member resigning, not being able, not being willing to continue their tasks as a Member, or being successfully impeached, the \hyperref[USGParliamentDef]{USG Parliament} will appoint an interim for the remaining period of the resigning member. For resignations and impeachments the following rules apply: 
	
	\protect\begin{parenum}
		\item Any elected member of the USG may resign provided that the USG is informed two (2) weeks before the intended date of resignation.		
		\item Any member of the USG shall be impeached by either an absolute 2/3 majority vote within the Parliament or the respective Committee, or a referendum. Such a vote or referendum must clearly state the grounds for impeachment. Repeated unexcused absence of three (3)  meetings or more and/or not accomplishing designated tasks automatically start the process of impeachment of a Committee or Parliament member:
	The accused member and the impeacher (if any) have one (1) week time to present their case.
	At the next \hyperref[USGParliamentDef]{USG Parliament} meeting, the impeachment is voted upon.		
	\item After resignation or impeachment, an officer may not run again for the same office within the term that the resignation or impeachment took place.
	\end{parenum}

\section{Amendments to the Constitution}
Any changes to part \ref{PartA}, that is article 1--\ref{art:elect}, as well as article \ref{art:final} are treated as changes to the constitution and are amended according to section \ref{sec:amend}.