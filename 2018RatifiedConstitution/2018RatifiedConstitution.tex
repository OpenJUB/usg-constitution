\documentclass[12pt]{LaTeX_Misc/constitution}
\usepackage{mathpazo}

\usepackage[T1]{fontenc}

\usepackage[tocflat]{tocstyle}
\usetocstyle{standard}

\usepackage{multicol}
\setlength\columnsep{30pt}

% Set options for enumerate environment
\usepackage{enumitem}
\setenumerate{nosep}

% Redefinition of ToC command to get centered heading
\makeatletter
\renewcommand\tableofcontents{%
  \null\hfill\textbf{\Large\contentsname}\hfill\null\par
  \@mkboth{\MakeUppercase\contentsname}{\MakeUppercase\contentsname}%
  \@starttoc{toc}%
}
\makeatother

% Renew paragraph command to give §1, §2, etc.
\renewcommand{\theparagraph}{\S\arabic{paragraph}}
\setcounter{secnumdepth}{4}

%\titleformat{\paragraph}[runin]{\bffont}{\S\theparagraph }{0em}{}[]

% Page numbering i, ii, iii for toc
\pagenumbering{roman}

\usepackage{hyperref}



\begin{document}


\begin{figure}[H]
    \centering
    \includegraphics[width=0.3\textwidth, right]{LaTeX_Misc/jacobs.png}
\end{figure}

\begin{figure}[H]
    \centering
    \includegraphics[width=0.3\textwidth]{LaTeX_Misc/usg.jpg}
\end{figure}

\begin{center}
\textbf{\Large{THE ESTABLISHMENT OF THE JACOBS UNIVERSITY\\
UNDERGRADUATE STUDENT BODY}}
\end{center}

\begingroup
\let\clearpage\relax
\setcounter{tocdepth}{2}

\renewcommand{\contentsname}{\centering Table of Contents}

\begin{multicols}{2}
\tableofcontents
\end{multicols}
\endgroup

\newpage


\linenumbers
\pagenumbering{arabic}
\setcounter{page}{1}

%==================================================
%                     Title 
%==================================================


\part{CONSTITUTION OF THE UNDERGRADUATE STUDENT BODY}


%==================================================
%                     Preamble 
%==================================================

\begin{center}
\textbf{Preamble}
\end{center}

{\parskip=\baselineskip
Acting as the voice of all students enrolled at Jacobs University, the Undergraduate Student Government (USG) \label{USGdef} shall actively work to improve the quality of life (in all its aspects) for the \hyperref[studentbody]{Undergraduate Student Body (USB)}, along with the university administration and community as a whole.

According to the \href{https://usg.jacobs.university/wp-content/uploads/2018/02/Academic_Constitution_2017_english.pdf}{Jacobs University Academic Constitution}, "the responsibilities of the Student Government include representing \textit{Students} towards the respective bodies within as well as outside Jacobs University, serving as the link between Students and university authorities, administration or other groups on campus and actively contributing to the communication between those bodies, appointing all representatives of students, and ensuring continuity."


Believing in the right of self-governance, the Undergraduate Student Body (USB) shall elect its Undergraduate Student Government (\hyperref[USGdef]{USG}), which is entirely student-run and politically independent of any administrative bodies. 
}

%==================================================
%                     USG Constitution
%==================================================

\label{PartA}

\article{The Undergraduate Student Government}
The \hyperref[USGdef]{USG} is formed in order to provide an official and representative organization to receive student questions and suggestions, investigate student problems and take appropriate action, provide the official voice through which the student opinion may be expressed, encourage the development of responsible student participation in the overall policy and decision making processes of the university community, foster an awareness of the students' role in the academic community, enhance the quality and scope of education at Jacobs University Bremen, provide means for responsible and effective participation in the organization of student affairs, and establish this Constitution for the Undergraduate Student Government at Jacobs University.


\section{Executive, Legislative and Judicial Power} 
The \hyperref[studentbody]{Undergraduate Student Body (USB)} shall have the right to self-governance through the Undergraduate Student Government (USG), which is the main authority body of the USB, through:
\begin{enumerate}
    \item the USG \hyperref[USGexecutiveUnitDef]{Executive Unit} (composed out of the \hyperref[USGstructure]{Student Committees}), having the executive power of the \hyperref[studentbody]{USB},
    \item the USG \hyperref[USGParliamentDef]{Parliament} \label{USGParliament} (composed out of the \hyperref[ElectedOfficesDef]{Elected Offices}), having the legislative power of the \hyperref[studentbody]{USB},
    \item the USG \hyperref[StudentCourtDef]{Student Court}, having the judicial power.
\end{enumerate}


\section{Power and Duty}
The \hyperref[USGdef]{USG} shall: 
\begin{enumerate}
\item inform the Student Body on recent, present and future endeavours,
\item organize activities involving the \hyperref[studentbody]{USB} and maintain the student budget,
\item establish student committees to deal with specific subjects, define their mandates, appoint their members, and monitor their activities,
\item meet regularly to discuss progress in student committees and to address all issues on the respective agenda,
\item at all times uphold and represent the interests of the \hyperref[studentbody]{USB},
\item enact legislation governing the conduct of the \hyperref[studentbody]{USB} after consultation with the \hyperref[studentbody]{USB},
\item serve as liaison with the university administration, faculty, staff and college authorities.
\end{enumerate}

\section{Student Rights}
Each student shall have the right to:
\begin{enumerate}[nosep] 
\item
vote in popular elections,

\item
run for elections or appointments for the offices of the \hyperref[USGdef]{USG}. This includes students currently a part of the \hyperref[USGdef]{USG} at the time of the elections, and excludes students on disciplinary probation,

\item
apply to be part of a committee,

\item
affiliate themselves with a college of her/his choice, in case of living off-campus,

\item 
affiliate themselves with a focus area of her/his choice, for Foundation Year students,

\item
start a student club and to join any club (regardless of background), according to the requirements set by the Campus Life and the by-laws of the \hyperref[USGdef]{USG},

\item
initiate a popular initiative or petition to suggest legislative proposals or any kind of action to the \hyperref[USGdef]{USG}.
\end{enumerate}

\section{Student Duties}
Each student shall have the duty and obligation to:
\begin{enumerate}
\item
act in accordance with the provisions stipulated by the constitution and the active by-laws that support it,

\item 
accept final decisions of the Disciplinary Board,

\item
 respect decisions taken by the \hyperref[USGdef]{USG} as long as they are in accordance with this constitution.
\end{enumerate}


%==================================================
%                     Executive Unit
%==================================================


\article{The Undergraduate Student Executive Unit} 
\label{USGexecutiveUnitDef}


\section{The Undergraduate Student Executive Unit Structure} 
\label{USGstructure}
The Undergraduate Student Executive Unit Structure shall have the following committees and departments:
\begin{enumerate}
\item \hyperref[PresidentOfficeDef]{President's Office}, consisting of the \hyperref[PresDef]{President} and the \hyperref[VPdef]{Vice-President},
\item \hyperref[AACdef]{Academic Affairs}, chaired by a \hyperref[FARepDef]{Focus Area Representative},
\item \hyperref[CACdef]{Campus Affairs}, chaired by a \hyperref[CRepDef]{Residential College Representative},
\item \hyperref[Financesdef]{Financial Affairs}, chaired by the \hyperref[TreasurerDef]{Treasurer},
\item \hyperref[suppstrucdef]{Support structures}: Secretaries, IT Department, PR.
\end{enumerate}
The chairs for hyperref[AACdef]{Academic} and \hyperref[CACdef]{Campus Affairs Committees} are elected with a simple majority within the \hyperref[USGParliamentDef]{USG Parliament}.


\section{The President's Office} 
\label{PresidentOfficeDef}
The President's Office shall be the main representative figure of the \hyperref[studentbody]{USB} and the USG and be responsible for the coordination of all activities of the USG and its departments, having the following recurring tasks:
\begin{enumerate}
\item preside over meetings of the \hyperref[USGParliamentDef]{USG Parliament},
\item inform the \hyperref[studentbody]{USB} and the USG about matters falling within their competence, especially when requested by these bodies,
\item be the main responsible of the USG's actions and oversee its structure and inner workings,
\item oversee the USG's public relations externally, and coordinate the USG's presentation in the media, both electronically and printed,
\item supervise the appointment and/or selection of the supportive positions and infrastructure.
\end{enumerate}
It defines its own \hyperref[PresByLawsDef]{Bylaws}.

\section{The Academic Affairs Committee}
\label{AACdef}
The Academic Affairs Committee (AAC) shall be responsible for the representation of the undergraduate \hyperref[studentbody]{USB} in all Academic University Committees and serve as a point of contact for students regarding all academic matters, having the following recurring tasks:
\begin{enumerate}
\item evaluate academic policies,
\item give feedback about academic matters to faculty and the administration,
\item meet with the university leadership at least once a semester to discuss academic developments,
\item be represented by students of all focus areas,
\item regularly update the \hyperref[studentbody]{USB} and the President about its workings as well as the meetings with the university leadership and the Academic University Committees,
\item be responsible for meeting with the Leadership at least once per semester,
\item be the link between the \hyperref[studentbody]{USB} and leadership,
\item represent the \hyperref[studentbody]{USB}'s concerns and opinions in any relevant internal discussions.
\end{enumerate}
It defines its own \hyperref[AACByLawsdef]{Bylaws}.

\section{The Campus Affairs Committee} 
\label{CACdef}
The Campus Affairs Committee (CAC) shall be responsible for the representation of the \hyperref[studentbody]{USB} in all university-enacted-structures that affect student life. Its mission is to ameliorate student experience while at Jacobs University and to serve as a point of contact for students regarding all non-academic matters while having the following recurring tasks:

\begin{enumerate}
\item evaluate campus life policies give feedback about non-academic matters to CampusLife,
\item be responsible for meeting university leadership at least once a semester to discuss non-academic developments,
\item be represented by students of all residences,
\item regularly update the \hyperref[studentbody]{USB} and the President about its workings as well as about meetings with CampusLife and the university leadership,
\item oversee and foster the student clubs and events,
\item be the link between the \hyperref[studentbody]{USB} and the administration for internal issues of all kinds.
\end{enumerate}
It defines its own \hyperref[CACByLawsdef]{Bylaws}.

\section{The Financial Affairs Committee}
\label{Financesdef}
The Financial Affairs Committee shall be responsible for all USG matters pertaining to finances and strive to direct the USG budget in ways to benefit the USB, having the following recurring tasks:
\begin{enumerate} 
\item review and filter the incoming financial applications,
\item seek the approval of the:
\begin{enumerate}
    \item President`s Office for spendings exceeding $\euro$ 250,
    \item \hyperref[USGParliamentDef]{USG Parliament} for spending exceeding $\euro$500
\end{enumerate}
\end{enumerate}
It defines its own \hyperref[FinByLawsdef]{Bylaws}.

\section{Support structures} 
\label{suppstrucdef}
The Support structures shall foster the inner workings of the USG and help both the \hyperref[USGParliamentDef]{USG Parliament} and Executive Unit in their actions. Further roles can be defined and must be publicly justiifed, then they can be attributed to USG members depending on the respective need, in addition to the support structures explicitly listed below:
\begin{enumerate}[label={\textbf{\S\arabic*}}]
\item The Secretaries shall:
\begin{enumerate}
\item set the Agenda for USG meetings,
\item be responsible of communication and documents, such as agenda and minutes,
\item organize tasks, check for completion;
\end{enumerate}

\item The IT Department shall:
\begin{enumerate}
\item be responsible for the maintenance of the USG related websites and for the updated information,
\item be the link between the \hyperref[studentbody]{USB} and the IT department of the University.
\end{enumerate}

\item Public Relations Manager shall:
\begin{enumerate}
    \item be responsible for the image of the USG, and implicitly of the University, 
    \item override the President's internal and external communication power.
\end{enumerate}

\end{enumerate}
These roles shall be positions any student can apply for, and the USG shall elect a candidate, that is, the only USG members shall be eligible to vote.



\article{USG Parliament}]
\label{USGParliamentDef}
The USG Parliament shall have the role to coordinate the USG, to explicitly define its task and to distribute workload. The \hyperref[USGexecutiveUnitDef]{USG Executive Unit} shall fulfill the proposals of the USG Parliament, upon common agreement between the two bodies.

\section{Structure and Elected Offices} 
\label{ElectedOfficesDef}
The Undergraduate Parliament shall be composed of the Elected Offices:  
\begin{multicols}{2}
\begin{enumerate}
\item President
\item Vice-President
\item \label{FARepDef} Focus Area Representative: Diversity
\item Focus Area Representative: Health
\item Focus Area Representative: Mobility
\item \label{CRepDef}College Representative: Krupp 
\item College Representative: Mercator 
\item College Representative: College 3
\item College Representative: Nordmetall
\item \label{TreasurerDef} Treasurer
\end{enumerate}
\end{multicols}

\section{Elections}
\label{electiontimes}
The USG shall hold a general election twice a year, in particular:
\begin{enumerate}

\item in first week of December for: Vice-President, Focus Area representatives, Treasurer (and for any open position),
\item in first week of May for: President, College representatives (and for any open position) .
\end{enumerate}


\section{Term length}
By default, all elected positions' terms within the USG shall be one year long.
\begin{enumerate}[label={\textbf{\S\arabic*}}]
\item Exceptions of the above are:
\begin{enumerate}
\item 
the candidates are third year students elected in the December election of their third year, and they are graduating at the end of their 6\textsuperscript{th} semester,
\item 
the candidates are second year students elected in the December election of their second year, and they are going abroad during their 5\textsuperscript{th} semester,
\item
the candidates are running for an open position from a previous one-semester term,
\item
elected members choose to have a one semester term.
\end{enumerate}
\item
The current members at the time of the election will remain in office until the end of the respective semester. After the election results, the newly elected members are invited as standing guests in the meetings. 
\item
A student can run for a position as many times as they wish.

\end{enumerate}


%==================================================
%                     Student Court 
%==================================================

\article{Student Court}
\label{StudentCourtDef}
The \hyperref[StudentCourtDef]{Student Court} shall be an independent body, responsible of checking and judging the legitimacy of the actions of the \hyperref[USGParliamentDef]{USG Parliament} and Executive Unit. As it is may not have regular tasks, it shall be formed upon request by either the \hyperref[studentbody]{USB} via a petition, the \hyperref[USGParliamentDef]{USG Parliament} or the \hyperref[USGexecutiveUnitDef]{USG Executive Unit}, in order to prosecute an accused body.


\section{Power and Duties}
The \hyperref[StudentCourtDef]{Student Court} shall:
\begin{enumerate}
\item at all points of time uphold this constitution in the interests of the USB,
\item interpret this constitution, by-laws, amendments and all subsequent legislation in the spirit
of community standards,
\item mediate between two (2) or more parties of the USG or \hyperref[studentbody]{USB}, at the request of any of
the members involved,
\item determine in a trial, as specified in the by-laws on Student Court procedure, the guilt or
innocence of any object of its jurisdiction who is accused of violations of this constitution
and its by-laws, and assess penalties for such violations according to the by-laws on
Student Court procedure,
\item  be permitted to defer a case to the next higher instance as specified in the by-laws on
Student Court procedure, if the case is decided to be beyond its competence to judge,
\item review the decisions made and actions taken by the Student Parliament and ensure that
they are made under the provisions of this constitution and its by-laws,
\item issue an evaluation of the Student Parliament's budget allocation, efficiency and
progress on agenda points with recommendations for improvement, whenever asked to do so
\item recommend, at its discretion, changes in or additions to this constitution, its by-laws, or
subsequent legislation,
\item be present at official Student Parliament meetings if requested by the Student Parliament
or Government. 
\end{enumerate}

\section{Student Court Formation}
The \hyperref[StudentCourtDef]{Student Court} shall consist of five (5) judges directly elected.
\begin{enumerate}[label={\textbf{\S\arabic*}}]
    \item In case the \hyperref[USGParliamentDef]{USG Parliament} is accused of wrong doings, then the \hyperref[USGexecutiveUnitDef]{USG Executive Unit} takes the task of forming the Student Court. Three (3) Judges shall be members of the \hyperref[USGexecutiveUnitDef]{USG Executive Unit}, and two (2) judges shall be elected from the USB.
    
    \item In case the \hyperref[USGexecutiveUnitDef]{USG Executive Unit} is accused of wrong doings, then the \hyperref[USGParliamentDef]{USG Parliament} takes the task of forming the Student Court. Three (3) Judges shall be members of the \hyperref[USGParliamentDef]{USG Parliament}, and two (2) judges shall be elected from the USB.
    
    \item In case both the \hyperref[USGParliamentDef]{USG Parliament} and Executive Unit are accused of wrong doings, then the initiators of the petition against the two bodies takes the task of forming the Student Court. In this case, all five (5) judges shall be elected from the USB.
\end{enumerate}



\section{Student Court Procedure}
The \hyperref[StudentCourtDef]{Student Court} shall form within a week of the accusation and shall take the responsibility to reach a decision within two weeks after its formation. 
\begin{enumerate}[label={\textbf{\S\arabic*}}]
    \item A quorum for the meetings shall consist of three (3) Student Court judges and decisions shall be made by simple majority vote.
    \item Each judge shall have one vote.
\end{enumerate}


\section{Conflicts of interest}
All Student Court judges must declare all possible conflicts of interest to Student Court prior to any hearing. In the interest of impartiality, any judge that has a conflict of interest and also if a judge is involved in a trial before Student Court as the complainant or as the accused, the respective judge shall remove themselves from the case and be replaced.


\article{Student Representation}

The USG shall designate the \hyperref[USRdef]{Undergraduate Student Representatives (USRs)} in all official matters, according to Jacobs University`s Constitution. The USRs bring the perspective of Undergraduate Students to the Academic Board as laid out in the Academic Constitution of Jacobs University Bremen. 

\section{Committees with USB Representation}
Permanent committees where there are Undergraduate Student Representatives are:
\begin{enumerate}
\item Academic Senate (AS), as Young Learner,
\item University Education Committee (UEC),
\item Quality Management (QM),
\item Hiring Committees.
\end{enumerate}

\section{Decision-making}
The USRs shall be free to vote in the committees as their consciousness dictates. The Student Parliament may however  dictate the USRs vote in their committees, if a two-thirds (2/3) majority of all members of the Parliament or respective Committee decides so. If unable to attend the official meetings, the USR is responsible for finding substitutes.

\section{Young Learner}
The USR of the Academic Senate, henceforth named Young Learner, has the following fixed and recurring tasks: 
\begin{enumerate}
\item
regular attendance of all Academic Senate meetings,

\item
regular attendance of USG meetings of the: Parliament, Academic Affairs Committee,

\item
present a summary of the Academic Senate to the USG, after each AS meeting or whenever asked to do so by the \hyperref[USGParliamentDef]{USG Parliament}.
\end{enumerate}

\section{University Education Committee Representative}
The AAC chair shall become the USR for and thus a member of the University Education Committee (UEC) for the respective academic year. The AAC chair shall:
\begin{enumerate}
\item
attend all UEC meetings,

\item
at their discretion invite additional members of the AAC to attend UEC meetings,

\item
present a summary of the UEC to the USG, after each UEC meeting or whenever asked by the \hyperref[USGParliamentDef]{USG Parliament}.
\end{enumerate}

\section{Quality Management Representative}
The USR of the Quality Management (QM) Committee shall be elected withing the AAC and has the tasks to: 
\begin{enumerate}
\item
attend all QM Committee meetings, 

\item
present a summary of the QM Committee to the AAC after each meeting, and to the entire USG whenever asked to do so by the USG Parliament.
\end{enumerate}


\section{Election and Term of Office of the Young Learner}

The Young Learner elections take place at the beginning of the academic year, after the USG Parliament is established.
\begin{enumerate}[label={\textbf{\S\arabic*}}]
\item 
The current Young Learner, together with the the USG Parliament, shall be responsible for  the election of the Young Learner. 

\item 
The Young Learner is to be elected by all members (committees and Parliament) of the USG by a simple majority.

\item
The elected Young Learner takes office immediately for the respective academic year.

\item
Eligible candidates are all USG members, subject to candidate eligibility policies in the provisions.

\end{enumerate}



\section{Hiring Committees}
USRs in and thus members of Hiring Committees (HC) are selected by the respective HC chair, and not by students.


%==================================================
%                    COMMITTEE BYLAWS  
%==================================================

\part{COMMITTEE BYLAWS}
\label{PartB}

\article{General Provisions}

\section{Decision-making} 
A simple majority among the committee members is required to reach a decision within any committee. 

\section{Membership}
Next to the Committee chair, who has to be an elected member of the Parliament, the Committee consists of regular members, which are either:
\begin{enumerate}
    \item active: mandatory attendance at meetings and with voting rights within the Committee and the USG
    \item passive: are part of the mailing lists and can join the meetings, but do not have the active member voting rights
\end{enumerate}
Members are selected after an application process to the respective Committee chair.
\begin{enumerate}[label={\textbf{\S\arabic*}}]
\item The chair's  nominees must then be formally approved by the respective Committee. 
\item Memberships continue until disenrollment, unless the member resigns or is inactive (as decided by the USG Parliament).
\end{enumerate}


\section{Impeachment}
In case of a failure of any committee member (including the chair) to live up to his/her responsibilities with regards to the committee, the USG Parliament shall take appropriate actions, ranging from a reprimand to impeachment and/or a redelegation of tasks.

{\parindent=1em
After a successful impeachment of a committee member, an additional member shall be elected or appointed as soon as possible to fill the open position.}

\section{Responsibilities}
Any changes related to responsibilities, tasks and duties of the committee as well as procedures and guidelines within the committee that are of a permanent nature must be approved by the USG Parliament by a simple majority, and those of temporary nature are at the discretion of the chair.

\section{Duties and Powers of the Chair}
The chair takes full accountability and responsibility for the work of and within the committee and, thus, it shall be the duty of the chair to ensure that all committee tasks are being duly implemented. Each committee defines additional duties.

\begin{enumerate}[label={\textbf{\S\arabic*}}]
\item Communication with the USG Parliament. The chair shall represent and communicate relevant projects and suggestions of the committee work to the USG Parliament and vice versa. 

\item The chair shall preside over the meetings of his/her committee.

\item At the beginning of each year, the chair must present a project list for the respective task and how to accomplish them. The chair shall write committee reports if required by the \hyperref[USGdef]{USG}.

\item The chair shall provide general advice concerning the committee's work for the succeeding committee chair, specifying the activities which were pursued and completed, stating any pending activities, and including suggestions for improved performance in the committee in the next academic year. The chair shall ensure that all relevant documents and materials are handed over to the succeeding committee chair. 

\item Establishment of the committee. Upon assignment of the committee the chair shall nominate committee members. 

\item It is at the discretion of the chair to invite additional members to attend their respective official meetings. If unable to attend, the chair is responsible for finding substitutes.

\item Representation to the Academic Senate (AS). The chair of any \hyperref[USGdef]{USG} committee shall have the right to be part of the USG delegation sent to regular student representative meetings with the AS, unless the USG Parliament decides otherwise.
\end{enumerate}


\article{Academic Affairs Committee By-Laws}
\label{AACByLawsdef}
\section{General Responsibilities of the Committee}
The Academic Affairs Committee (AAC) of the Undergraduate Student Government (\hyperref[USGdef]{USG}) shall gather undergraduate student opinions on current academic issues and represent and mediate the students' opinions towards faculty and staff. 

\begin{enumerate}[label={\textbf{\S\arabic*}}]
\item 
The committee shall strive for the best solution for the \hyperref[studentbody]{USB} as a whole. It is the objective of the committee to enhance overall teaching quality.

\item The committee shall be the contact institution for students who have problems with or suggestions about academic policies and regulations. The committee shall maintain an open mind and look for solutions and new ideas within the academic realm at Jacobs University.
\end{enumerate}

\section{Fixed and/or recurrent tasks}

The committee shall work on relevant suggestions and requests from individual students or other institutions. The fixed and recurrent tasks of the committee shall pertain to:

\begin{enumerate}

\item 
any academic issue brought up in the UEC, by the USG or the USB,

\item
all issues relevant to education, that is, teaching, learning and assessment, 

\item 
structure changes and their implications,

\item
major evaluations,

\item
awards, such as the Teacher of the Year and Lecture Hall Awards.
\end{enumerate}

\section{Duties of the chair}
The AAC chair shall become the undergraduate student representative and a member to the UEC for the respective academic year. The chair shall attend all UEC meetings. It is at the discretion of the chair to invite additional members of the AAC to attend UEC meetings.

\article{Campus Affairs Committee By-Laws}
\label{CACByLawsdef}

\section{Duties of the Campus Affairs Committee}
\textbf{\S{}1} The primary purpose of the Campus Affairs Committee (CAC)  is to facilitate and improve life on campus for students. The CAC shall consider and discuss with the appropriate university authorities any relevant issue for which there is a proven student interest, as long as the area of interest does not infringe upon the areas supervised by other committees, determined by either of:
\begin{enumerate}
\item the committee and/or the parliament's discretion,
\item any student petition that has more than 50 signatures.
\end{enumerate}

{\parskip=\baselineskip
\parindent=1em
\textbf{\S{}2} Any student shall have the right to approach the CAC for initiation of the proceedings described above upon consideration of the \hyperref[USGParliamentDef]{USG Parliament}.
The CAC shall have the right to file an official complaint to Campus Life against any employee on campus provided that there exists evidence beyond reasonable doubt on the issue, and with an absolute majority vote of the Parliament.}


\section{Student Clubs}
Any student shall have the right to ask the \hyperref[USGParliamentDef]{USG Parliament} to check the respective club`s inner workings. \textit{Only for clubs that receive USG funding,} any club member shall have the right to ask the USG to impeach the respective club`s president and hold club elections. 
\begin{enumerate}[label={\textbf{\S\arabic*}}]
\item Should an impeachment proceeding occur, the decision shall be decided by a simple majority vote of the USG Parliamentarians.
\item Should a member of the \hyperref[USGdef]{USG} be involved in the impeachment procedure (accused or member of the club), he/she shall not have the right to vote on the impeachment for the sake of impartiality.
\end{enumerate}


\article{Financial Affairs Committee By-Laws}

%%%%%%%%%%%%%%%%%%%
%% SECTIONS OF THIS ARTICLE
%% 1. General Responsibility (i.e. Philosophy)
%% 2. Tasks
%% 3. Duties of Treasurer
%% 4. Funding Regulations (i.e. specific stuff)

%% GET RID OF EVERYTHING ELSE

\label{FinByLawsdef}
\section{General Responsibilities of the Committee}
The Financial Affairs Committee (FAC) of the Undergraduate Student Government (USG) shall manage the USG Budget. The USG and FAC shall, with their available funds, endeavor to financially support public, non-profit events, initiatives and the like of students, which are to the potential benefit of the student body as a whole. These include events on campus, competitions attended by a delegation of the student body, student-led projects and student-led excursions. These are only examples; general rules are that the initiative shall:
\begin{enumerate}
\item	not be exclusive but potentially open to any student to attend or be a part of, 
\item	be carried out to the advancement of the student body and the benefit in improving student life in academic, socio-cultural or other aspects,
\item	not be harmful to Jacobs University's image or reputation.
\end{enumerate}

Neither the FAC, nor the USG shall financially assist individual students. 

 The Chair of the FAC shall have direct contact with the Campus Life Department regarding the undergraduate students' funds and the USG Budget. % and lawful distribution/accounting and whatnot (the university _should_ have no authority over OUR money. Legally, you`re right, but because we are not a legal entity and unable to have continuation, they do it for us.


\section{Fixed and Recurring Task}
The fixed tasks of the FAC shall be: 
\begin{enumerate}
\item to convene weekly in meetings chaired by the Treasurer,
\item the evaluation of Funding Applications or Addendums,
\item the approval or rejection of a Funding Application or an Addendum
\item the writing of confidential, USG-internal reports on why respective Funding Applications or Addendums were approved or rejected, containing specific recommendations for the USG Parliament,
\item the submission of approved Funding Applications and Addendums as well as the corresponding reports to the USG Parliament,
\item the checking and approval of receipts and invoices and their submission of such receipts and invoices to the Campus Life Department,
\item the accounting of the USG Budget and notification of the USG Parliament about the situation of the USG Budget,
\item communication to the respective staff within the administrative body of Jacobs University.
\end{enumerate}

\section{Duties of the Treasurer}
The FAC shall be led by an USG Parliamentarian who has been directly elected in the position of ``Treasurer'' by the student body. The Treasurer takes the reponsibility for:

\begin{enumerate}
\item all work of and within the Financial Affairs Committee,
\item attendance of all USG Parliament meetings and presentation of current FAC matters,
\item communication with the USG such as recommendations or reports on Funding Applications,
\item the submission of an annual budget report at the end of each academic year to the USG Parliament,
\item passing on all relevant information to the succeeding Treasurer,
\item the establishment of the FAC as defined in Article \ref{FinByLawsdef}, Section \ref{fin:establishment}.
\end{enumerate}

\section{Establishment of the FAC}\label{fin:establishment}
{\parindent=1em
\textbf{\S{}1} The Treasurer shall conduct a formal application procedure in conjunction with a second USG Parliamentarian for students wishing to join the FAC.

\textbf{\S{}2} The size of the FAC for effective and efficient functioning shall be decided by the Treasurer. 

\textbf{\S{}3} Committee members of the FAC shall hereafter be referred to as ``Financial Experts''.}


\section{Student's Right to Apply for Funding}
All students of Jacobs University shall have the right to apply for funding of the USG for their initiatives and/or club activities regardless of age, gender, race, religion, major or year of study, as long as the criteria specified in these By-Laws are met.

\section{Substantial Funding Applications}
The FAC has the authority to fund Applications or Addendums that amount up to $\euro$250 and the obligation to emit a formal notification about such funding to the USG Parliament. Any amount exceeding $\euro$250 will require an approval by the President's office, and any amount exceeding $\euro$500 will require vote of approval by a simple majority in the USG Parliament.

\section{Profit-Generating Events}
The FAC shall only fund profit-generating events, initiatives, and the like as long as they meet one of the following criteria:
\begin{enumerate}
    \item profits are reinvested in the continuity of the club or event,
	\item profits are donated to a cause deemed appropriate by the Parliamentary Finance Committee if the Funding Application is $\euro$250 or less, and by the Undergraduate Student Government if the application exceeds $\euro$500,
	\item profits are returned to the USG Budget.
\end{enumerate}

\section{Miscellaneous Funding Regulation}
{\parindent=1em
\textbf{\S{}1} 
% SPECIFIC: REDISTRIBUTION (not to be in this documnet)
The FAC has the authority to approve or reject Addendums for redistribution of funds, as long as they remain in the limit of the initial budget. Addendums intent on raising the initial budget stipulated in the initial Funding Applications can only be approved by the FAC if they request up to $\euro$200, as otherwise they have to be approved by the USG Parliament.

% SPECIFIC: LOGO FOR ORGANIZATIONS (requirement not there yet in any previous section!!!)
\textbf{\S{}2} The FAC has the authority to waive the requirement according to which the organizers are obliged to portray the USG logo in certain events or clubs.

% SPECIFIC: LENDING PROPERTY ?!
\textbf{\S{}3} The FAC has the authority to lend out USG property with reasonable assurance of return. The Chair is responsible for the management of property loans. If the Chair shall be indisposed for an extended period of time, the USG Parliament can opt for ceding this power to the USG President for the duration of the Chair's absence if the Chair has provided a signed temporary removal of power.}


\article{President's Office}
\label{PresByLawsDef}
\section{General Responsibilities of the President's Office}
The President's Office and its Support Structures shall convene and preside weekly meetings of the USG Parliament. The President's Office shall oversee the workings of the committees of the USG, taking the information and their opinions into account when making decisions.

\section{Duties of the President}
\label{PresDef}
The President shall act as the primary representative of the USG to the \hyperref[studentbody]{USB}, and the primary representative of the USB to the University Administration.
The President shall ensure the proper functioning and efficiency of the Committees of the USG.

%%% TODO EXTERNAL POWER

\section{Duties of the Vice President}
\label{VPDef}
The Vice President shall aid the President in any matters pertaining to the duties of the President to the best interest of the USG. The Vice President shall further:
\begin{enumerate}
\item \label{VPdeputy} be the Acting Head of the USG whenever the President is absent and/or not legitimized,
\item oversee the matters of the Disciplinary Council and attend the hearings when they occur. \item be the foremost contact for further developing and advancing the disciplinary procedures at Jacobs University together with its administration.
\end{enumerate}


\section{Event of Untimely Termination}
In the event that the President/Vice-President steps down from his/her role before the proper conclusion of the term, a new President shall be elected from the existing members of the \hyperref[USGParliamentDef]{USG Parliament} in the earliest possible convenience. 
\begin{enumerate}[label={\textbf{\S\arabic*}}]
\item In accordance with Article \ref{PresByLawsDef}, Section \ref{VPDef}, Point \ref{VPdeputy}, the Vice President shall become the Interim President of the USG until the proper introduction of the new President into his/her office.
\item There shall be no role of interim Vice President until the conclusion of the said election. 
\end{enumerate}



%==================================================
%                    PROVISIONS  
%==================================================



\part{PROVISIONS}
\label{PartC}

\article{Definitions}

\section{Constitution}
Unless stated otherwise, the term "constitution" refers to the text of the present constitution, as legislative document of the Undergraduate Student Body (USB).
 
\section{Students and Undergraduate Student body (USB)}
\label{studentbody}
 Unless stated otherwise, the term "students" refers to undergraduate students as defined in the university constitution, i.e. Bachelor and pre-degree students currently enrolled at Jacobs University. Unless stated otherwise, the term "student body" refers to the Undergraduate Student Assembly, as defined by Jacobs University constitution.

\section{USG Member}
Any elected position or member of a committee of the USG is an USG member.

\section{Undergraduate Student Representative (USR)}
\label{USRdef}
The Undergraduate Student Representative (USR) is the person physically participating at the University meeting, as stipulated by the Jacobs University Constitution.


\section{Provision}
A provision is a complementary law adopted by the Student Government according the constitution. They shall be referenced in the text of the constitution as soon as they are adopted. They are approved or amended by a simple majority vote.

\section{Student club}
A student club is a student-run organization focused on a social/academic/sport or leisure purpose. To be official, it has to be registered with Campus Life.

\section{Student budget}
The student budget refers to the aggregate amount of money paid by students as the USG tax collected by the university.

\section{Popular elections}
All elections in the \hyperref[studentbody]{USB} shall be general, direct, free, fair, secret and held accordingly to the Election Procedure By-Laws of Article \ref{electionprocedurebylaws}.

\section{Referendum}
\label{referendum}
A referendum is a legislative act of the USG decided upon by a popular vote of the \hyperref[studentbody]{USB}. The outcome of a referendum shall be binding on the USG.
\begin{enumerate}[label={\textbf{\S\arabic*}}] 
\item
The organization of a referendum requires a simple majority of all Members of the USG to pass it to a student vote. 
\item
A referendum is considered to be successful if a simple majority votes in favour, with a threshold quorum of a fifth (20\%) of all eligible voters.
\item
No decisions shall be implemented if a referendum is pending on an issue.
\end{enumerate}

\section{Student Opinion}
\label{StudentOpiniondef}
The USG may consult the \hyperref[studentbody]{USB} for future decisions via the Student Opinion on the \href{https://vote.jacobs.university/}{voting platform}. Its purpose is to give more information and leverage to the actions of the USG. However, the outcome of the Student Opinion shall be binding on the USG, unless the \hyperref[USGParliamentDef]{USG Parliament} votes in absolute unanimity against it.

\section{Popular initiative} 
 A popular initiative is an issue, not voted on by the USG, relating to the \hyperref[studentbody]{USB}. It is put directly to a popular vote by the \hyperref[studentbody]{USB} without prior consultation with the USG. The outcome of the popular initiative shall be binding on the USG. The organization of popular initiatives requires a petition signed by at least twenty (5\%)  of all students. The Members of the USG shall approve the initiative or add options to the ballot to ensure feasibility with an absolute majority vote. Ballot options included on the petition for the popular initiative may only be excluded by the Members of the USG for reasons of feasibility.

\section{General Meeting of the USG} 
 A general meeting is an official meeting called for by the USG to which all students are invited.

\section{Absolute and Simple Majority}
An absolute majority refers to a voting outcome in which more than fifty percent (50\%) of the total number of the respective constituency (present or not) votes in favor.  A simple majority refers to a voting outcome in which there are more "in favor" votes than "against" votes among the physically present members; abstentions are part of the pool.

\section{Absolute and Simple Unanimity}
An absolute unanimity refers to a voting outcome in which all members of the respective constituency (present or not) votes in favor. A simple unanimity refers to a voting outcome in which all physically present members vote in favour.


\section{Campaign Period}
The period from announcing a candidature and before the voting for it starts shall be known as the campaign period.

\section{Impeachment}
The process of removing someone from their office.


\article{Election Procedures}
\label{electionprocedurebylaws}

The provisions of this article shall only apply to elections for positions in the USG.

\section{Election Committee} 
The election committee (EC) consists of the elected members of the USG who do not finish their term at the respective election time. If requested, they shall be supported by any remaining members. The election committee has to form itself at the president`s initiative. The committee then elects a head of elections. 

\section{Special Elections} 
Any other elections (except the two specified by the constitution in \ref{electiontimes}) during the academic year shall be organized by the President of the USG who chairs the respective election committee.

\section{Conflict of Interest}
Each parliament and secretariat member shall have the duty to report any irregularities to \hyperref[StudentCourtDef]{Student Court}. No member of the student court, parliament and secretariat who is supervising the election shall be a candidate in or initiator of the election in question.

\section{Rules of Procedure} 
The election is carried out in accordance with the provisions set forth in the Constitution of the \hyperref[studentbody]{USB} and these provisions.

\section{Candidates' Applications}
Students wishing to apply shall submit their application to the EC, which will collect the applications and prepare them for the election. Applications may consist of a letter of intent (one-pager), explaining the motivation for applying, past experience in student governments or university administration, an outline of an agenda they plan to pursue and proposed solutions, a strategy for engaging with the USG and other stakehoders.


\section{Election Announcements} 
All elections shall be announced no later than two (2) days prior to the election, naming the options or candidates and functions they are running for, place, date, and time of the election. Such announcements shall take place through a notification sent to all student email accounts and by other platforms set by the USG via the EC.

\section{Candidates' Eligibility}
The student shall be physically present in the respective semester of being in the office.
\begin{enumerate}[label={\textbf{\S\arabic*}}]
\item
Students in their third year of study may not run for an office in the USG in the May elections of their last year.

\item
Students in their second year of study who will spend the fifth semester abroad or will do an internship in their fifth semester may not run for an office in the USG in the May elections of their second year.

\item
Foundation Year students may only run for an office if they have been offered admission for undergraduate studies at Jacobs University and have accepted that offer.

\item 
Only in exceptional cases, as decided by the \hyperref[USGParliamentDef]{USG Parliament}, one student is allowed to be the Undergraduate Student Representatives of more than one official committees of Jacobs University with student representatives (USR).

\end{enumerate}

\section{Announcing the Candidates}
The EC is responsible for officially announcing the candidates.
\begin{enumerate}[label={\textbf{\S\arabic*}}]
\item
Candidates in or initiators of an election shall be allowed to have an official public profile using an online forum set-up by the EC, in order to present their ideas and opinions relevant to the election.
\item
A General Assembly (GA) organized by the EC will take place during the campaigning period, i.e. after the candidates announcement and before the voting starts.
\item
The purpose of the GA is to (physically) present the candidates, and to have a debate forum where any member of the Jacobs Community shall have the opportunity to ask questions, especially towards the candidates.
\end{enumerate}

\section{Campaigning}
Candidates shall have the right to democratically campaign in an appropriate and honest way. The Election Committee shall have the responsibility to more explicitly define the above, if requested by any student. All candidates agree and accept that:
\begin{enumerate}[nosep] 
\item
campaigning is only allowed during the campaign period,
\item
slander is strictly prohibited,
\item
in the case that candidates by their own wish shall need financial resources for campaigning, no funding from USG or Campus Life shall be used.
\end{enumerate}


\section{Candidates' Misconduct} 
The election committee shall have the right to establish additional regulations regarding public campaigning if deemed appropriate and impose sanctions/disqualify candidates. These additional regulations must comply with the Constitution of the \hyperref[studentbody]{USB}.

%\section{Time and Place of Elections} 
%All elections shall take place in the college serveries or a similar adequate room that is to be determined by the election committee, during lunch and dinner on the day(s) of the election.

\section{Voter Eligibility} 
Every member of the \hyperref[studentbody]{USB} who has not lost his or her right to vote shall be eligible to vote in any election by the \hyperref[studentbody]{USB}. 

\section{Online Ballots} 
The IT Department has the task to set up an election platform for online voting. Ballots have to be approved by the election committee prior to the election. To be considered fit for being used in voting, a ballot has to contain all options presented in an equal way without discriminating any of them.

\section{Violations} 
Violations of the election rules and regulations set forth in these by-laws may result in:
\begin{enumerate}[nosep]
\item the violator losing his or her status as a candidate in the election
\item the violator losing his or her status as eligible voter for future elections
\item the election being invalid
\item new election results being issued.
\end{enumerate}
Such cases shall be handled and decided upon by the EC.

\section{Publication of results}
Immediately after the vote-tallying is finished, the election committee shall inform the secretariat about the results of the election. The secretariat shall then publish the results of the election, including the number of eligible voters, the number of votes cast, the number of valid votes, the number of invalid votes, the number of votes for each of the options or candidates, the number of votes for each of the options or candidates per college, and naming the winning option or candidate. This shall be done through a notification sent to all student email accounts and an announcement posted on the student forum.

\section{Taking Office}
Once the newly-elected members of the \hyperref[USGParliamentDef]{USG Parliament} are determined, the newly-elect members of the \hyperref[USGParliamentDef]{USG Parliament} and the outgoing members take responsibility that the newly-elected members of the \hyperref[USGParliamentDef]{USG Parliament} are trained in their respective tasks.

\article{Meeting Policies and Student Voice}


\section{Chair and Quorum}
The chair of the \hyperref[USGParliamentDef]{USG Parliament} meetings is the President, and if missing, the Vice-President. A meeting consists of at least a quorum of four members of the \hyperref[USGParliamentDef]{USG Parliament} and the President or the Vice-President.

\section{Regular Meetings}
Meetings have to take place regularly and are open to the general public of Jacobs University, unless the Student Parliament or any Committee decides with a simple majority to close the meeting. 
\begin{enumerate}[label={\textbf{\S\arabic*}}]
\item Closed meetings are attended by elected or appointed members only. 
\item The USG Parliament may choose to include any other student if such is considered appropriate by a simple majority. 
\item Attendance for Members of the \hyperref[USGParliamentDef]{USG Parliament} is mandatory.
\item The meetings have to be made public at least two days prior to the meeting.
\end{enumerate}


\section{Decision-Making} 
Each member of the USG shall have one (1) vote in all matters. Decisions shall be made by a simple majority vote with at least fifty percent (50\%) of all members voting. Voting takes place publicly and the names of the "ayes", "nays", and "abstentions" shall be recorded in the minutes. With a simple majority vote the USG may declare a secret vote (ballot) in which case only the number of votes shall be recorded. 


\section{Student Vote}
The USG may ask the \hyperref[studentbody]{USB} via a \hyperref[StudentOpiniondef]{Student Body Opinion} procedure regarding USG decisions and future actions. However, the Student Body Opinion may be overruled with a full unanimity of the \hyperref[USGParliamentDef]{USG Parliament}.


\section{External Outreach Power}
The USG shall have the right to communicate with external parties, as stated by Academic Constitution of Jacobs University Bremen. 
\begin{enumerate}[label={\textbf{\S\arabic*}}]
\item The USG members, as any student, shall have the right to contact any external bodies, only if they do so as individuals and not representing the USG.
\item The whole USG (Parliament and Executive Unit) has to be informed of any intention for external communication.
\item A simple majority vote in the \hyperref[USGParliamentDef]{USG Parliament}, among the USG members or a Student Opinion Decision may enact or prevent any external communication.
\item In particular, the USG President is the main responsible for any external communication.
\end{enumerate}

\section{Communication Responsibilities}
One of the USG`s core missions is to inform the \hyperref[studentbody]{USB}. 
\begin{enumerate}[label={\textbf{\S\arabic*}}]
    \item A regular task shall be a monthly newsletter of one page with USG past, current and future projects relevant to the respective time-frame.
    \item Failure to do so for two months consecutively constitutes a valid reason for impeachment.
\end{enumerate}

\section{Conflicts of Interest}
Conflicts of interest must be declared and the involved members must be omitted from voting. In case of a tie in the voting procedure, the final decision is made jointly by the President, Vice-President and the respective Committee Head.



\section{Resignation and Impeachment}
In the event of an USG Member resigning, not being able, not being willing to continue their tasks as a Member, or being successfully impeached, the \hyperref[USGParliamentDef]{USG Parliament} will appoint an interim for the remaining period of the resigning member.

\begin{enumerate}[label={\textbf{\S\arabic*}}]
\item  Any Member of the USG may resign provided that the USG is informed two (2) weeks before the intended date of resignation.

\item Any member of the USG shall be impeached by either of the two: Two thirds (2/3) of the (other) elected members or referendum. Such a vote or referendum must clearly state the grounds for impeachment. Repeated absence (more than thee (3) in a row) and/or not accomplishing designated tasks automatically start the process of impeachment of a member:
\begin{enumerate}
\item The accused member and the impeacher (if any) have one (1) week time to argument and present their case.
\item At the next \hyperref[USGParliamentDef]{USG Parliament}  meeting, the impeachment is voted upon.
\end{enumerate}

\item After resignation or impeachment, an officer may not run again for the same office within the term that the resignation was submitted in. 
\end{enumerate}



\article{Final provisions and policy changes}\label{art:final}

\section{Constitution Scope}
The constitution of the \hyperref[studentbody]{USB} at Jacobs University is of a general character. Any foregoing article of the constitution may be complemented, according to this constitution, with provisions issued by the USG.
\begin{enumerate}[label={\textbf{\S\arabic*}}]
\item \hyperref[PartA]{Part A} of this document is the explicit Constitution, that has to be amended according to Article \ref{art:final}, Section \ref{Ammendments}.

\item \hyperref[PartB]{Part B} represents By-Laws and secondary provisions, that can be changed internally by the \hyperref[USGParliamentDef]{USG Parliament} according to Article \ref{art:final}, Section \ref{sec:internal-changes}.

\item \hyperref[PartC]{Part C} represents complementary provisions, definitions and guidelines.

\end{enumerate}

\section{Amendments and Referendum}
\label{Ammendments}
Amendments of this constitution or the ratification of a new constitution can only be passed in a successful referendum by the \hyperref[studentbody]{USB}. Before the organization of a referendum, any new constitution or amendments to the old constitution must be introduced via a general assembly to the \hyperref[studentbody]{USB}. 

\section{Overruling} 
The provisions of this constitution shall not violate in any way general provisions of German, European, or international law as far as applicable. If such provisions exist, they shall be immediately revoked and/or amended.

\section{Internal Changes and Amendments of By-Laws}\label{sec:internal-changes}
\textbf{\S 1} Existing by-laws and provisions can be amended or replaced and new ones can be introduced only with a two-thirds (2/3) majority of members voting in favour.

{\parindent=1em
\textbf{\S 2} If the name of a document, committee, unit, or office mentioned in this constitution changes without the change of the respective entity violating this constitution, this document may be updated by substitutions of the old name with the new name upon approval of the USG by a simple majority vote.}


\section{Announcing changes}
If additional provisions are issued in form of complementary by-laws, the publicly available text of this constitution must clearly mention the names and articles of the relevant by-laws.


\section{Termination}
This constitution shall expire upon ratification of a new constitution by the \hyperref[studentbody]{USB}.



\vfill
\nolinenumbers
\emph{Version 1.0 of this constitution is effective as ratified by the student body on February 28th, 2018.}\\

Contact for questions and suggestions:
\renewcommand{\labelitemi}{--}
\begin{itemize}[nosep]
\item Daniel Prelipcean <\url{d.prelipcean@jacobs-university.de}> (AAC Chair 2017-2018)
\item Marco David <\url{m.david@jacobs-university.de}> (AAC Chair 2018)
\end{itemize}

\end{document}
